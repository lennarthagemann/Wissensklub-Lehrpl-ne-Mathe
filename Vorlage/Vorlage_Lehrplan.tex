\documentclass{article}
\usepackage[ngerman]{babel}
\usepackage[margin=2cm]{geometry}
\usepackage{amsmath, amssymb, amsfonts}
\usepackage{tikz}
\usepackage{graphicx}
\usepackage{fancyhdr}

\graphicspath{{../}}
%Seitenstil modifizieren
\pagestyle{fancy}% eigenen Seitestil aktivieren}
\fancyhf{}% Alle Felder loeschen

\fancyhead[L]{
\begin{tabular}[b]{l}
Lernstandsübersicht\\
Mathematik\\
Sekundarstufe XY
\end{tabular}}
\fancyhead[R]{
\includegraphics[height=3\baselineskip]{Wissensklub-Logo.png}
}
\addtolength{\headheight}{2\baselineskip}
\addtolength{\headheight}{0.61pt}
\fancyhead[C]{2024}

\begin{document}
\begin{titlepage}
    \begin{center}
        \vspace*{1cm}
            
        \Huge
        \textbf{Lernstandsübersicht Klasse 5}            
        \vspace{0.5cm}
        \LARGE
        Mathematik
            
        \vspace{1.5cm}
            
        \textbf{Wissensklub GmbH}
            
        \vfill
            
        Sekundarstufe XY\\
        \textit{Schulform} Nordrhein-Westfalen
            
        \vspace{0.8cm}
            
        \includegraphics[width=0.5\textwidth]{Wissensklub-Logo.png}
            
        \Large
        2024          
    \end{center}
\end{titlepage}
\section{Zahlen und Größen}
\subsection{Inhalte}
An dieser Stelle werden exemplarisch die Inhalte des Themas kurz vorgestellt
\subsection{Beispiele}
\subsection{Fragen}
Hier werden Aufgaben leichten bis mittleren Schwierigkeitsgrads vorgestellt. Diese sind zur Veranschaulichung  des Themas gedacht. 
\newpage
\section{Symmetrie}
\subsection{Inhalte}
\subsection{Beispiele}
\subsection{Fragen}
\newpage
\section{Rechnen}
\subsection{Inhalte}
\subsection{Beispiele}
\subsection{Fragen}
\newpage
\section*{Leistungsüberprüfung}
Hier werden zu allen Themen zusätzliche Fragen gestellt um den aktuellen Wissensstand der Schüler*innen festzustellen.
\end{document}