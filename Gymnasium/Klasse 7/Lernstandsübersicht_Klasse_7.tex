\documentclass{article}
\usepackage[ngerman]{babel}
\usepackage[margin=2cm]{geometry}
\usepackage{amsmath, amssymb, amsfonts}
\usepackage{tikz}
\usepackage{graphicx}
\usepackage{fancyhdr}

\usepackage{tcolorbox} %Umgebung mit farbigem Hintergrund

\graphicspath{{../../}}
%Seitenstil modifizieren
\pagestyle{fancy}% eigenen Seitestil aktivieren}
\fancyhf{}% Alle Felder loeschen

\fancyhead[L]{
\begin{tabular}[b]{l}
Lernstandsübersicht\\
Mathematik\\
Klasse 7
\end{tabular}}
\fancyhead[R]{
\includegraphics[height=3\baselineskip]{Wissensklub-Logo.png}
}
\addtolength{\headheight}{2\baselineskip}
\addtolength{\headheight}{0.61pt}
\fancyhead[C]{2024}

\begin{document}
\begin{titlepage}
    \begin{center}
        \vspace*{1cm}
            
        \Huge
        \textbf{Lernstandsübersicht}\\            
        \vspace{0.5cm}
        \LARGE
        Mathematik
            
        \vspace{1.5cm}
            
        \textbf{Wissensklub GmbH}
            
        \vfill
            
        Klasse 7\\
        \textit{Gymnasium} Nordrhein-Westfalen
            
        \vspace{0.8cm}
            
        \includegraphics[width=0.5\textwidth]{Wissensklub-Logo.png}
            
        \Large
        2024          
    \end{center}
\end{titlepage}

\section{Rechnen mit rationalen Zahlen}
In diesem Thema wird das Rechnen mit negativen Zahlen eingeführt und auf Brüche und Dezimalzahlen erweitert.
\textbf{Für dieses Thema sind aus der 6. Klasse die Themen \textit{Rechnen mit Brüchen und Dezimalzahlen}, \textit{Brüche vergleichen} \textit{Brüche auf dem Zahlenstrahl} und \textit{Negative Zahlen im Koordinatensystem} Vorraussetzung.}

\subsection{Inhalte}

\subsubsection*{Ganze Zahlen}
\textit{Was bedeutet eine negative Tordifferenz?} \\
Negative Zahlen sind aus dem erweitereten Koordinatensystem und dem Alltag, wie z.B. bei Temperaturen bekannt. Der Zahlenstrahl wird um negative Zahlen erweitert, jede Zahl bekommt eine Gegenzahl, die sich nur um das Vorzeichen unterscheidet..
Kinder sollen ganze Zahlen auf der Zahlengerade markieren können und nach der Größe ordnen können.
Es soll anhand der Zahlengerade vermittelt werden, was der Abstand zweier Zahlen sind, z.B. $-3$ und $14$ haben den Abstand $17$.


\subsubsection*{Rationale Zahlen und ihre Anordnung}
Wie bereits bekannt füllen die Brüche oder Dezimalzahlen die Lücken zwischen den ganzen Zahlen auf der Zahlengerade.
Auch Brüche bekommen nun ein Vorzeichen und sollen der Größe nach sortiert und eingezeichnet werden können.
Der Betrag einer Zahl, also der Abstand einer Zahl von der 0, wird eingeführt.
\subsubsection*{Positive Zahlen addieren und subtrahieren}
\textit{Wie verändert sich eine Temperatur über den Tag, welchen Wert haben die Sprünge?}\\
Anschaulich mit der Zahlengerade werden Subtraktion und Addition als Sprung auf der Gerade nach links oder rechts erklärt. Man startet bei einer bestimmten Zahl und geht dann Schritte im Wert des Betrages einer Zahl in eine entsprechende Richtung.
Es sollen SUmmen und Differenzen an der Zahlengerade berechnet werden, anschließend nur noch mittels der Zahlen. Auch ein Überschlag und ungefähres Rechnen wird erprobt.
\subsubsection*{Negative Zahlen addieren und subtrahiere}
Wenn negative Zahlen addiert oder subtrahiert werden, lässt sich eine Rechnung mit der Gegenzahl ausführen. Hier ändern sich die Vorzeichen, je nachdem  ob $++$. $+-$, etc. auftaucht.
Wie bei positiven Zahlen wird dies mit und ohne Zahlenstrahl, genau oder mit Überschlag gerechnet.
Zahlen können in Dezimalschreibweise oder als bruch vorkommen.
\subsubsection*{Multiplizieren und Dividieren rationaler Zahlen}
Die Regeln zum Vorzeichenwechsel für Multiplikation und Division werden aus der wiederholten Addition und dann der Multiplikation mit dem Kehrwert hergeleitet.
Beim Überschlagen und ausrechnen längerer Terme sollen die Rechengesetze geschickt verwendet werden.
\subsubsection*{Rechenvorteile nutzen}
\textit{Wie kann ich schneller als der Taschenrechner sein?}\\
Mit dem Kommutativ-, Assoziativ- und Distributivgesetz bei Multiplikation und Division sollen Terme geschickt ausgerechnet werden.
Hier werden Minusklammern passend aufgelöst und praktische Zahlen durch Rechenvorteile in Zwischenschritten bestimmt.
Hier werden alle bekannten Zahlentypen wie gemischte Brüche mit Vorzeichen und Dezimalzahlen vermischt, und in den richtigen Schritten müssen Aufgaben nacheinander bearbeitet werden.
Das strukturierte Arbeiten ist hierbei die größte Herausforderung!
\subsection{Beispiele}
\begin{tcolorbox}[colback=gray!5!white,colframe=gray!25!black]
Test
\end{tcolorbox}
\subsection{Fragen}
\begin{tcolorbox}[colback=blue!5!white,colframe=blue!25!black]
Test
\end{tcolorbox}
\newpage

\section{Zuordnungen}
Zwischen zwei Größen werden Zusammenhänge als Zuordnungen behandelt, insbesondere proportionale und antiproportionale Zuordnungen.
\textbf{Wichtig für dieses Kapitel sind aus der 5. und 6. Klasse {Dreisatz}, \textit{Koordinatensysteme}, \textit{Rechnen mit Einheiten}, \textit{Terme mit einer Variablen} und \textit{Rechnen mit Brüchen und Dezimalzahlen}}.
\subsection{Inhalte}
\subsubsection*{Zuordnungen darstellen}
Bekannte Zuordnungen aus dem Alltag wie Temperatur $\leftrightarrow$ Uhrzeit und Geschwindigkeit oder Distanz $\leftrightarrow$ Zeit werden vorgestellt und in Diagrammen  durch Punktpaare festgehalten.
Aus einem Graphen soll eine Zuordnung beschrieben werden. Aus einer Zuordnung sollen aus einer Vielzahl an Graphen die richtige identifiziert werden.
\subsubsection*{Zuordnungen mit Formeln beschreiben}
\textit{Wie weit ist der Sturm entfernt anhand von der Zeit zwischen Donner und Blitz?}
Häufig ist ein rechnerischer, z.B. linearer Zusammenhang bei einer Zuordnung festzustellen.
Dann können wir einen Term mit einer Variablen angeben, um diese Zuordnung zu beschreiben.
Kinder sollen mit diesen Termen Wertetabellen füllen können. Es werden Zuordnungen den passenden Graphen zugeordnet.
Der Term wird genutzt um beliebige Wertepaare auszurechnen.
\subsubsection*{Proportionale Zuordnungen}
Bei einer proportionalen Zuordnung verändern sich beide Werte im gleichen Maß: Verdoppelt sich ein Wert so verdoppelt sich auch der andere Wert.
Es wird der Proportionalitätsfaktor eingeführt, mit welchem man einen Term für die Zuordnung aufstellen kann.
Es werden Graphen aus proportionalen Zuordnungen hergestellt. Es muss im Sachkontext begründet werden, ob eine Zuordnung proportional ist.
Der Dreisatz wird genutzt um den Proportionalitätsfaktor und andere Werte aus der Reihe zu ermitteln.

\subsubsection*{Antiproportionale Zuordnungen}
Wenn sich ein Wert des Paares vervielfacht, wird bei einer antiproportionalen Zuordnung der andere Wert um den gleichen Faktor geteilt.
Hier findet man eine Antiproportionalitätskonstante als Produkt eines beliebigen Wertepaares der Zuordnung.
Antiproportionale Zuordnungen sollen im Sachkontext identifiziert und als Wertetabelle und Graph dargestellt werden.
Mit dem Dreisatz sollen weitere Zahlenpaare gefunden werden.
\subsubsection*{Zuordnungstypen erkennen und nutzen}
Es werden Techniken zum Identifizieren einer Zuordnung vermittelt. Bei einer gegebenen Wertetabelle soll durch den Dreisatz eine (Anti-)Proportionalitätskonstante gefunden werden können.
Aus dem Sachkontext soll eine Abhängigkeit festgestellt und die passende Zuordnung bestimmt werden.
\subsection{Beispiele}
\begin{tcolorbox}[colback=gray!5!white,colframe=gray!25!black]
Test
\end{tcolorbox}
\subsection{Fragen}
\begin{tcolorbox}[colback=blue!5!white,colframe=blue!25!black]
Test
\end{tcolorbox}
\newpage

\section{Prozentrechnung - Zinsrechnung}
Die Prozentrechnung wird weiter in Rechnungen genutzt und die Brücke zu Zinsen und Zinseszinsen geschlagen.
\textbf{Für dieses Kapitel sind das Thema \textit{Brüche} und \textit{Prozente}, \textit{relative Häufigkeiten}, \textit{Dreisatz} und \textit{Das Rechnen mit Brüchen und Dezimalzahlen} Vorraussetzung. Für Zinsen und Zinseszinsen ist das Rechnen mit \textit{Potenzen} zentral.}
\subsection{Inhalte}
\subsubsection*{Prozentsätze berechnen}
Im Kontext der Prozentrechnung werden für Brüche die Begriffe Grundwert, Prozentwert und Prozentsatz eingeführt.
Hier wird vor allem im Sachkontext von Geld gerechnet, z.B.  ein Rabatt oder die prozentuelle Preissenkung. 
Die neuen Begriffe sollen errechnet werden können und die Rolle der Wahl des Grundwertes verdeutlich werden.
\subsubsection*{Prozentwerte berechnen}
Wenn der Grundwert und Prozentsatz bekannt sind, dann wird Prozentwert anders ausgerechnet durch das Umstellen der Formel für den Prozentsatz.
Alternativ kann mit dem Dreisatz und dem bekannten Grundwert, also $100\%$, ein Prozentwert für einen beliebigen Prozentsatz ermittelt werden.
Auch hier können Überschlagsrechnungen für ungefähre Ergebnisse genutzt werden.
\subsubsection*{Grundwerte berechnen}
Wennn nun der Prozentsatz und Prozentwert bekannt sind, kann entweder wieder durch das Umstellen der Formel oder durch den Dreisatz der Grundwert ermittelt werden.
Mögliche Aufgaben lassen sich wie in den beiden vorigen Themen stellen. \\
Schließlich sollen sich die Kinder die Zusammenhänge aller drei Werte je nachdem was bekannt ist bewusst werden, um alle fehlenden Werte berechnen zu können.
Wichtig ist hierbei die Identifikation von Grundwert, Prozentwert und Prozentsatz, insbesondere im Sachkontext.
\subsubsection*{Zinsen}
Für den Kontext von Zinsen werden die drei neuen Begriffe umgemünzt zu Kapital, Zinssatz und Zinsen.
Diese drei Begriffe sollen mit Grundwert, Prozentsatz und Prozentwert verknüpft werden, um Zinsen für Erträge oder Schulden zu berechnen, aber genauso auch Startguthaben oder Zinssätze.
\textit{Zinseszinsen}
Hier werden Zinsen über einen längeren Zeitraum berechnet. Der Zinsfaktor wird eingefügt, um das Rechnen mit Prozenten zu vereinfachen.
Es wird eine Formel hergeleitet für das Errechnen von Zinsen über einen beliebigen Zeitraum.
Es sollen Zinseszinsen ausgerechnet werden, Zeiträume um einen bestimmten Wert zu erreichen. Auch das Arbeiten mit Excel kann hier erprobt werden.
\subsubsection*{Dezimalzahlen Mulitiplizieren und Dividieren}
Für dieses Thema ist das Konzept der Kommaverschiebung der Schlüssel.
Kinder sollen Dezimalzahlen Mulitiplizieren indem sie zuerst das Komma ignorieren, und anschließend anhand der Summe der Nachkommastellen der Faktoren im Ergebnis das Komma an die richtige Stelle setzen.
Dadurch sollen Kinder auch das Ergebnis weiterer Aufgaben, in denen das Komma an einer anderen Stelle steht aber die Ziffern gleich sind, ohne Rechnung aufschreiben können.
Bei der Division müssen sich die Kinder bewusst werden, das wenn man bei Dividend und Divisor das Komma gleich verschiebt, das Ergebnis gleich bleibt, z.B. $200 : 50 = 20:5 = 2:0.5 = 0.02 : 0.005 = 4$.
Dann verändern die Kinder den Term so, dass der Divisor eine natüürliche Zahl ist und beachten das Komma beim Endergebnis.
Hier sollen auch wieder die Rechengesetze genutzt werden, um schwierige Aufgaben in der bestmöglichen Reihenfolge zu rechnen.
Hier lernen die Kinder das erste Mal das Distributivgesetz und die Konzepte Ausmultiiplizieren und Ausklammern kennen.
\subsection{Beispiele}
\begin{tcolorbox}[colback=gray!5!white,colframe=gray!25!black]
Test
\end{tcolorbox}
\subsection{Fragen}
\begin{tcolorbox}[colback=blue!5!white,colframe=blue!25!black]
Test
\end{tcolorbox}
\newpage

\section{Terme und Gleichungen}
Terme werden um Variablen ergänzt und vereinfacht. Lineare Gleichungen werden aufgestellt und gelöst, um damit bekannt Probleme aus dem Alltag zu modellieren.
\textbf{Hier sind die vorigen Themen \textit{Terme}, \textit{Rechengesetze} und \textit{Zuordnungen} Vorraussetzung! Für Bruchterme und Bruchgleichungen ist ein sehr gutes Verständnis der Bruchrechnung notwendig.}
\subsection{Inhalte}
\subsubsection*{Terme mit einer Variablen}
Zuordnungen die mit einer Rechenformel bzw. Term ausgerechnet werden, werden als Begriff zu Termen mit einer Variable verallgemeinert.
Variablen sind Teile eines Terms in den beliebige Zahlen eingesetzt werden können, damit kann ein neuer Wert des Terms ausgerechnet werden.
Es sollen Werte von Termen für gegebene Zahlen ausgerechnet werden können. Dies kann in einer Wertetabelle festgehalten werden.
Weiterhin soll aus einem Zusammenhang, wie z.B. einer Zuordnung ein Term aufgestellt werden.
\subsubsection*{Terme mit einer Variablen umformen}
Terme können durch das Umstellen (Kommutativgesetz), Ändern der Reihenfolge (Assoziativgesetz) und Zusammenrechnen gemeinsamer Faktoren (Distributivgesetz) vereinfacht werden.
Hier ist wichtig das nur Summanden mit und nur Summanden ohne Variablen jeweils zusammengefasst werden können. 
Mit einem Rechenweg soll eine Umformung anhand der Rechengesetze begründet werden.
Terme können zunächst aus einem Kontext hergeleitet und dann vereinfacht werden.
Es kann überprüft werden ob zwei Terme gleichwertig sind.

\subsubsection*{Ausmultiplizieren und Ausklammern}
Das Distributivgesetz wird für Terme mit Variable eingesetzt. Dabei sollen schrittweise bekannte Regeln wie das Ausklammern oder die Minusklammerregel und das Distributivgesetz mit Division genutzt werden.
Damit können Terme umgeformt werden und die Gleichwertigkeit begründet werden.
\subsubsection*{Gleichungen aufstellen und lösen}
Es wurden Terme mit einer Variable ausgerechnet für gegebene Werte: Das Ergebnis ist gleichwertig mit dem eingesetzten Term. 
Eine Gleichung besteht aus zwei gleichwertigen Termen und einem Gleichheitszeichen.
Es soll vermittelt werden, das Gleichungen sich besser eignen um einen Wert für einen Term zu finden als systematisches Ausprobieren verschiedener Zahlen.
Es wird die Technik des \textit{Rückwärtsrechnen} erlernt um Gleichungen zu lösen. Eine Probe wird genutzt um die eigene Rechnung zu überprüfen.

\subsubsection*{Gleichungen mit Äquivalenzumformungen lösen}
Äquivalenzumformungen, wie Rückwärtsrechnen, können genutzt werden um Gleichungen zu lösen.
Hier wird genutzt das man auf beiden Seiten die gleiche Rechnung macht, da sich dadurch an der Gleichheit beider Seiten nichts ändert.
Es sollen Strategien zum Finden der passenden Äquivalenzumformung gefunden werden, um eine Gleichung in wenigen Schritten zu lösen.
\subsubsection*{Bruchterme und Bruchgleichungen}
Es werden Gleichungen mit rationalen Zahlen eingeführt, in denen also Bruchterme vorkommen können.
Der Lösungsansatz mit Äquivalenzumformungen bleibt dabei gleich, wird aber ergänzt um die Kompetenz zum Rechnen mit Brüchen.
\subsubsection*{Problemlösen mit Gleichungen}
Gleichungen sind ein praktisches Werkzeug um Phänomene in der echten Welt zu modellieren.
Es wird geschult Probleme strukturiert in seine Bausteine zu zerlegen. Was ist gesucht, was ist gegeben und wichtig? Was ist meine Variable, wie sieht der Term aus? Wie löse ich die Gleichung? Was bedeutet das Ergebnis für die Aufgabe? Kann mein Ergebnis stimmen?
Gegebenenfalls können auch Ungleichungen eingeführt werden. Es sollen Probleme im Sachkontext gelöst werden, und Einheiten sinnvoll genutzt werden.
Probleme müssen im Kontext beanwortet werden, eine Rechnung alleine ist nicht ausreichend.
\subsection{Beispiele}
\begin{tcolorbox}[colback=gray!5!white,colframe=gray!25!black]
Test
\end{tcolorbox}
\subsection{Fragen}
\begin{tcolorbox}[colback=blue!5!white,colframe=blue!25!black]
Test
\end{tcolorbox}
\newpage

\section{Konstruieren und Argumentieren}
Winkel in verschiedenen Formen werden verglichen und Zusammenhänge werden erkannt. Kongruenzsätze werden hergeleitet und benutzt.
\textbf{Vorraussetzung ist aus der 6. Klasse \textit{Winkel}, \textit{Winkel messen und zeichnen}, \textit{Kreise} und ein guter Umgang mit dem Geodreick und Zirkel.}
\subsection{Inhalte}
\subsubsection*{Winkel an sich schneidenden Geraden}
Bei zwei oder mehr sich schneidenden Geraden entstehen ähnliche Winkel. Nebenwinkel, Scheitelwinkel, Stufenwinkel und Wechselwinkel werden als Begriffe eingeführt.
Diese werden genutzt um die Größe unbekannter Winkel mit Termen und Gleichungen herzuleiten.
Es sollen hiermit Winkelgrößen ohne zu Messen bestimmt werden können. Auch bei Vielecken, z.B. Trapezen werden solche Winkel identifiziert und zum Finden fehlender Werte genutzt. 

\subsubsection*{Winkelsummen}
Bei vielen Figuren vom gleichen Typ, z.B. Dreiecke und Vierecke haben die Innenwinkel die gleiche Summe.
Es wird der Innenwinkelsatz für Dreiecke genutzt,  um fehlende Winkel im Dreieck rechnerisch zu ermitteln.
Dies wird kombiniert mit Dreiecken die sich aus dem Schnitt mehrerer Geraden ableiten um die Konzepte ähnlicher Winkel miteinfließen zu lassen.


\subsubsection*{Dreiecke konstruieren}
Dreiecke können aus unvollständigen Informationen hergeleitet werden.
Dazu reicht z.B. eine Seitenlänge und zwei Winkel, oder drei Seitenlängen.
Hier werden die Bezeichnung anhand der Planfigur eines Dreiecks für die Seiten-, Ecken- und Winkelnnamen eingeführt.
Je nach Information muss eine Konstruktion mit Geodreieck und/oder Zirkel gemacht werden.
Im Sachkontext können Aufgaben wie z.B. die Vermessung großer Gebäude oder Licht- und Schattenfiguren gefragt werden um Dreieckskonstruktionen zu nutzen.
\subsubsection*{Kongruenz}
Dreiecke sind kongruent, wenn man sie vollständig übereinander ohne Überstand legen könnte. Hierzu reichen z.B. die Informationen über alle Seitenlängen, zwei Seitenlängen und einem Winkel, \ldots.
Verschiedene Dreiecke sollen auf Kongruenz überprüft werden. Dabei muss der verwendete Kongruenzsatz angegeben werden.
Es muss entschieden werden können, ob die Informationen reichen um auf Kongruenz zu überprüfen.
\subsubsection*{Mit Kongruenzsätzen Argumentieren}
Mittels der Kongruenzsätze können die ersten mathematischen Beweise geführt werden. Hier handelt es sich um Beweise z.B.  zu den Seitenlängen bei gleischenkligen Dreiecken. Es sollen die Grundbausteine eines Beweises verstanden, insbesondere Vorraussetzung, Definitionen und Behauptung.
\newpage
\section{Wahrscheinlichkeiten}
Der Zufall wird mit Wahrscheinlichkeiten mathmetisch bearbeitet. Zufallsexperimente werden beschrieben und Wahrscheinlichkeiten geschätzt oder mit Baumdiagrammen genau ausgerechnet.
\textbf{Vorraussetzung ist sein sehr guter Umgang mit \textit{Brüchen}, \textit{Prozenten}, \textit{relative und absolute Häufigkeiten} und das \textit{Multiplizieren von Brüchen}}
\subsubsection*{Wahrscheinlichkeit}
Es werden Zufallsexperimente, deren Ergebnis nicht genau vorhersehbar ist, eingeführt. Bei gegebenen Ergebnissen kann man absolute und relative Häufigkeiten zählen, jetzt benutzen wir vor der Ausführung von Experimenten unsere Erwartungen als  Wahrscheinlichkeit.
Wahrscheinlichkeiten beim Münz- oder Würfelwurf sollen geschätzt werden, analytisch oder experimentell.
Wahrscheinlichkeiten sollen aus relativen Häufigkeiten abgeleitet und in Prozent angegeben werden.

\subsubsection*{Laplace-Experimente}
Wenn alle Wahrscheinlichkeiten bei einem Zufallsexperiment gleich sind, nennen wir dieses Laplace-Experiment.
Ein Ereignis ist eine Zusammenfassung von mehreren Ergebnissen, z.B. ein gerader Würfelwurf besteht aus den Ereignissen $2,4,6$. Die Summenregel für Laplace-Experimente wird hergeleitet.
Diese soll genutzt werden um Wahrscheinlichkeiten von Ereignissen auszurechnen.
Dafür sollen Ergebnisse in Ereignisse zerlegt werden.
Aus Wahrscheinlichkeiten können die erwarteten absoluten Häufigkeiten ausgerechnet werden, je nachdem wie häufig ein Experiment wiederholt wird.
\subsubsection*{Baumdiagramme}
Experimente die in mehreren Schritten/Stufen ablaufen können mit Baumdiagrammen beschrieben werden.
Die Wahrscheinlichkeiten Multiplizieren sich entlang eines Pfades des Baumes.
Es sollen Wahrscheinlichkeiten der \textit{Blätter} an einem Baumdiagramm berechnet werden können. Ereignisse sollen mit ihren Ergebnisse am Baumdiagramm beschrieben werden.
Baumdiagramme sollen im Kontext erstellt werden oder ausgesucht werden können.
Es soll entschieden werden, ob ein Baumdiagramm angemessen ist um eine Aufgabe zu lösen.

\end{document}