\documentclass{article}
\usepackage[ngerman]{babel}
\usepackage[margin=2cm]{geometry}
\usepackage{amsmath, amssymb, amsfonts}
\usepackage{tikz}
\usepackage{graphicx}
\usepackage{fancyhdr}

\usepackage{tcolorbox} %Umgebung mit farbigem Hintergrund

\graphicspath{{../../}}
%Seitenstil modifizieren
\pagestyle{fancy}% eigenen Seitestil aktivieren}
\fancyhf{}% Alle Felder loeschen

\fancyhead[L]{
\begin{tabular}[b]{l}
Lernstandsübersicht\\
Mathematik\\
Sekundarstufe XY
\end{tabular}}
\fancyhead[R]{
\includegraphics[height=3\baselineskip]{Wissensklub-Logo.png}
}
\addtolength{\headheight}{2\baselineskip}
\addtolength{\headheight}{0.61pt}
\fancyhead[C]{2024}

\begin{document}
\begin{titlepage}
    \begin{center}
        \vspace*{1cm}
            
        \Huge
        \textbf{Lernstandsübersicht Klasse 5}\\            
        \vspace{0.5cm}
        \LARGE
        Mathematik
            
        \vspace{1.5cm}
            
        \textbf{Wissensklub GmbH}
            
        \vfill
            
        Klasse 5\\
        \textit{Gymnasium} Nordrhein-Westfalen
            
        \vspace{0.8cm}
            
        \includegraphics[width=0.5\textwidth]{Wissensklub-Logo.png}
            
        \Large
        2024          
    \end{center}
\end{titlepage}
\section{Zahlen und Größen}
In diesem Thema werden die Grundrechenarten und das Verständnis von Zahlen vertieft. Dabei wird erarbeitet, wie das Rechnen in unserem Zahlensystem funktioniert, und dieses Wissen auf das Rechnen mit verschiedenen Einheiten übertragen.
Dies dient zur Vorbereitung und Erlernen des Rechnen mit Kommazahlen / Dezimalzahlen.
\subsection{Inhalte}
\subsubsection*{Zählen und Darstellen}
\textit{Wie kann ich das Ergebnis einer Umfrage festhalten? } \\Hier werden Strichlisten, Tabellen und verschiedene Arten von Diagrammen wie zum Beispiel Säulendiagramme und Balkendiagramme vermittelt.
\subsubsection*{Zahlen Ordnen}
\textit{Wie können wir Zahlen auf einer Linie der Größe nach ordnen? Welche Hausnummer bekommt welches Haus? }\\
Bei \textit{Zahlen und Darstellen} werden Zahlen anhand von Kategorien in Diagrammen dargestellt. Hier werden Zahlen am Zahlenstrahl veranschaulicht, um sie der Größe nach zu sortieren.
Es wird ein Verständnis vermittelt wie man Zahlen unterschiedlicher Größe schnell miteinander vergleicht.
\subsubsection*{Runden}
\textit{Wie kann ich große Zahlen schnell einordnen und abschätzen?}\\
Große Zahlen können wir anhand ihrer Ziffern in Einer, Zehner, Hunderter, \ldots aufteilen. Außerdem werden für immer größere (längere) Zahlen neue Zahlwörter eingeführt: Tausender, Millionen, Milliarden, \ldots. Dabei lernen die Kinder das Dezimalsystem kennen.
Schließlich werden Zahlen auf- und abgerundet, je nachdem welche Zahl die Ziffer rechts von der Rundungsstelle beträgt.
\subsubsection*{Rechnen mit Geld, Längen Gewichten, Zeit, \ldots}
\textit{Wie kann ich berechnen wie viel der Eintritt in den Zoo für eine Gruppe kostet? Passen meine Möbel an eine Seite meines Zimmers?}\\
Hier wird insbesondere im Sachkontext das Rechnen mit Einheiten vermittelt. Zunächst werden die bekannten Stellenwerttafeln benutzt, danach wird auf Kommazahlen aufgebaut.
Zusätzlich kommt beim Rechnent mit Längen, Gewichten und Zeit das Umrechnen in ide nächstgrößere bzw. -kleinere Einheiten hinzu. Insbesondere bei diesem Thema spielen Sachaufgaben eine zentrale Rolle.
\subsection{Beispiele}
\subsection{Fragen}
Hier werden Aufgaben leichten bis mittleren Schwierigkeitsgrads vorgestellt. Diese sind zur Veranschaulichung  des Themas gedacht. 
\newpage
\section{Symmetrie}
\subsection{Inhalte}
\subsubsection*{Senkrechte und parallele Geraden}
\textit{Was ist der kürzeste Weg um über die Straße zu laufen?}\\
Mithile des Geodreiecks wird vermittelt, wie zwei Geraden bzw. Strecken zueinander stehen. Mit diesen Konzepten kann man sinnvoll die Strecke zwischen einem Punkt und einer Gerade sowie zweier Geraden messen. Zentral ist hierbei das saubere Zeichnen mithilfe des Geodreiecks.
\subsubsection*{Koordinatensystem}
\textit{Wie kannn ich bei Schiffe versenken die genaue Position ansagen?}\\
Um Zeichnungen im Heft und die Lage von Punkten und Strecken besser beschreiben zu können, nutzen wir ein Koordinatensystem. Wir beschreiben Punkte anhand ihrer $x$- und $y$- Koordinate.
Hier wird sich  spielerisch an das Zeichnen komplizierter Formen im Koordinatensystem herangetastet.
\subsubsection*{Achsensymmetrische Figuren}
\textit{Welche Gegenstände werden aus zwei spiegelbildlichen Hälften zusammengesetzt? Was ist die Symmetrie einer Windmühle?}\\
Mithilfe des Geodreiecks werden Symmetrieachsen herangeführt. Damit können Spiegelungsachsen gezogen werden, um Formen achsensymmetrisch zu spiegeln. 
Anschließend wird die Punktsymmetrie eingeführt, bei welche einer Figur anhand eines einzelnen Punktes gespiegelt wird.
\subsubsection*{Eigenschaften von Vielecken}
\textit{Welche Symmetrien können verschiedene Dreiecke haben? Welche Gemeinsamkeiten und Unterschieden haben besondere Vierecke?}\\
Anhand von Symmetriegeinschaften, Senkrechten und Parallelen sowie Seitenlängen können wir besondere Vielecke finden. Dazu gehören: rechtwinklige, gleichschenklige und gleichseitige Dreiecke, Quadrate, Rechtecke, Rauten, Trapeze, \ldots. Mithilfe der erlernten Symmetrieigenschaften sollen diese Formen möglichst geschickt mit dem Geodreieck gezeichnet werden können.
\subsection{Beispiele}
\subsection{Fragen}
\newpage
\section{Rechnen}
In diesem Kapitel wird das Rechnen von Zahlen mittels Gesetzen bzw. Vorteilen verallgemeinert. Dadurch können wir schriftliches Rechnen erlernen, mit Potenzen arbeiten und über Teilbarkeit und Primzahlen reden.\\
\textbf{Für dieses Kapitel ist ein sehr gutes Verständnis von Thema 1 nötig!}
\subsection{Inhalte}
\subsubsection*{Terme}
\textit{Wie kann ich Schritt für Schritt eine Rechnung erklären?}\\
Ein Term ist eine Folge von Rechenanweisungen. Wichtig ist hier die Reihenfolge von Rechenanweisungen. Diese ist entscheidend wenn Punkt- und Strichrechnung vermischt werden. Die Reihenfolge kann auch durch das Einsetzen von Klammern beeinflusst werden.
\subsubsection*{Rechengesetze(-vorteile)}
\textit{In welcher Reihenfolge kann ich einen Term am einfachsten ausrechnen?}\\
Bei manchen Termen können wir selber entscheiden in welcher Reihenfolge wir rechnen. Das Vertauschen von Summanden bei der Addition oder Faktoren bei der Multiplikation heißt Kommutativgesetz. Das Weglassen oder Setzen von Klammern bei einer Summer oder Produkt heißt Assoziativgesetz.
Durch geschicktes Einsetzen dieser Gesetze können leichte Rechnungen, z.B. wenn diese glatte Hunderter oder Zehner ergeben vorgezogen werden. \\
\textit{In welche und wie viele Teilgruppen gleicher Gröpe kann ich eine große Gruppe aufteilen?} \\
Zusätzlich wird mit dem Distributivgesetz das vermischte Rechnen von Addition und Multiplikation erleichtert. Das Distributivgesetz kann genutzt werden um eine schwere Multiplikation in leichtere aufzuteilen.
\subsubsection*{Potenzieren}
\textit{Wie dick wird ein Papier wenn ich es wiederholt falte?}\\
Ein Term in dem immer wieder der gleiche Faktor multipliziert wird dauert sehr lange aufzuschreiben. Eine kürzere Schreibweise ist die Potenzschreibweise, bestehend aus Grundzahl und Hochzahl. Kinder sollen Produkte in Potenzen und andersrum umwandeln können. Anschließend sollen Potenzen ausgerechnet und auch Terme die eine Potenz beinhalten ausgerechnet werden können.
\subsubsection*{Teilbarkeit}
\textit{Bleibt etwas übrig wenn jeder zwei Äpfel aus dem Obstkorb nimmt?}\\
Hier wird der Begriff des Vielfachen, Teilers und Teilbarkeit eingeführt. Für bestimmte Zahlen, wie z.B. die 2,5 und 10 können leichte Regeln zur Prüfen der Teilbarkeit erlernt werden.
\subsubsection*{Primzahlen und Primfaktorzerlegung}
Zahlen die keine weiteren Teiler außer sich selbst und die eins haben heißen Primzahlen. Jede Zahl kann in Primfaktoren zerlegt werden, also in eine Multiplikation von Primzahlen. Hier werden Strategien entwickelt um alle Teiler einer Zahl zu finden. Damit kannn auch überprüft werden, welche Zahlen prim sind oder nicht.
\subsubsection*{Schriftlich Rechnen}
\textit{Wie rechne ich schwere Rechnungen aus, die ich nicht im Kopf rechnen kann?}\\
Beim schriftlichen Rechnen werden die Zahlen stellengerecht aufgeschrieben und nach einem Schema augerechnet. In diesem Thema kommen sehr häufig Sachaufgaben vor, weswegen die Themen \textit{Rechnen mit Einheiten} und \textit{Runden} gut beherrscht werden müssen.
\subsection{Beispiele}
\subsection{Fragen}
\newpage
\section{Flächen}
In diesem Thema werden Flächeninhalte von bekannten geometrischen Formen und zusammengesetzten Figuren berechnet. Zusätzlich können Einheiten im Sachkontext ergänzt werden. Maßstäbe von Karten werden eingeführt, um Längen in der Wirklichkeit im Bild und in der Wirklichkeit ineinander umzurechnen.
\textbf{Hier sind die vorigen Themen \textit{Eigenschaften von Vielecken}, \textit{Koordinatensysteme}, \textit{Rechnen mit Einheiten} und der Umgang mit dem Geodreieck Vorraussetzung!}
\subsection{Inhalte}
\subsubsection*{Flächeninhalte vergleichen}
\textit{Kann ich zwei Flächen aus Legosteinen ineinander umlegen?}
Flächeninhalte werden durch die Anzahl ihrer Kästchen oder anderer kleiner Teilformen verglichen. Formen sollen von den Kindern zerteilt werden können in ihre \textit{Bausteine} und durch das Zählen verglichen werden. 
\subsubsection*{Flächeneinheiten}
\textit{Reicht der eimer Farbe zum Streichen der Wand?}
Anstelle von Kästchen, ist es allgemein verständlicher Flächeneinheiten zu benutzen. Die Umrechnung zwischen Flächeninhalten ist anders als bei Längeneinheiten und wird geprobt. Flächen können in gemischter Schreibweise mehrerer Einheiten angegeben werden.
\subsubsection*{Flächeninhalte}
\textit{Mit wie viel Holzbalken kann ich eine Terasse belegen?}
Es werden für das Rechteck und das rechtwinklige Dreieck Formeln zur Berechnung des Flächeninhalts ergründet. Kinder sollen diese Körper messen können und die Flächen mit Einheiten berechnen. Auch zusammengesetzte Körper sollen unterteilt und berechnet werden.
Hier bieten sich Sachaufgaben an, dies benötigt ein starkes geometrisches Vorstellungsvermögen.
\subsubsection*{Umfänge}
\textit{Wie lang ist ein Spaziergang um den See?}
Der Umfang ist die Länge aller Randstrecken eines Vielecks. Für Rechtecke und Quadrate gibt es leichte Formeln für den Umfang. Wichtig ist hier der Unterschied zwischen Flächen und Längen, und wie sich Flächen und Umfang verändern wenn wir die Seitenlängen anpassen.
\subsubsection*{Maßstäbe}
\textit{Eine Eiffelturmfigure ist 3000-Mal kleiner als der echte Eiffelturm. Wie hoch ist der Eiffelturm in Wirklichkeit?}
Ein Maßstab gib an, wie sich eine Länge in einer Abbildung zu der Länge in der Wirklichkeit verhält. Es müssen Strecken in Zeichnungen ausgemessen werden, wie z.B. Karten, und dann die tatsächliche Länge mit dem Maßstab ermittelt werden.
Alternativ soll anhand von Informationen über die echte Länge und einem Bild der Maßstab ermittel werden.
\subsection{Beispiele}
\subsection{Fragen}
\newpage
\section{Körper}
Es werden Eigenschaften von Körpern erarbeitet und benannt. Körper werden in Netze zerlegt sowie als Schrägbilder gezeichnet. Schließelich werden Längen Oberflächen und Rauminhalte bestimmt und Raumeinheiten zur Angabe von Volumina eingeführt.
Auch hier ist das Thema \textit{Rechnen mit Einheiten} Vorraussetzung und ein sehr guter Umgang mit dem Geodreieck.
\subsection{Inhalte}
\subsubsection*{ Körper und Netze}
\textit{Wie wird eine Pappchachtel gebaut und geklebt?}
Viele geometrische Körper kann man in Netze zerlegen. Dabei werden diese \textit{aufgeschnitten} und auf eine Fläche gelegt. Man kann sich diese Netze als Bastellvorlage für die jeweilige Form vorstellen.
Es sollen Körper erkannt werden, auch bei zusammengesetzen Figuren. Es werden die Seitenflächen nach Art und Form identifiziert. Netze sollen Körpern zugeordnet werden und andersrum. 
\subsubsection*{ Schrägbilder}
\textit{Aus welche Blickrichtung zeige ich einen Körper am besten auf ein Blatt Papier?}
Schrägbilder sind eine Art Körper zu zeichnen, ohne sie in ein Netz zu zerlegen. Es sollen Schrägbilder für Quader, Qürfel aber auch zusammengesetzter Formen angelegt werden.
\subsubsection*{ Rauminhalte vergleichen}
Wie bei der Zerlegung von Flächen in Quadrate, zerlegen wir Körper in gleichgroße Würfel. Durch das Zählen der Würfle können wir die Rauminhalte vergleichen und neue Körper zusammensetzen. 
\subsubsection*{ Volumeninhalte}
Über einen Würfel mit Kantenlänge $1m$ wird der Kubikmeter $m^3$ hergeleitet. Hier werden andere Volumeneinheiten abgeleitet und ineinander umgerechnet. Flüssigkeiten werden in Volumeneinheiten beschrieben, und Kinder sollen Volumen auch in gemischten Einheiten angeben können.
\subsubsection*{ Volumen eines Quaders}
\textit{Wie groß ist das Volumen eines Würfels aus Zuckerwürfeln?}
Für den Quader und den Würfel wird eine Vormel für das Volumen durch das Ausfüllen mit Würfeln hergeleitet. Die Herangehensweise das Volumen als Produkt von Grundfläche und Höhe zu berechnen wird alternativ vorgestellt.
Schließlich können so wieder zusammengesetzte Volumen berechnet werden.
\subsubsection*{Oberflächeninhalte von Quadern und Würfeln}
Bei Körpernetzen wird es greifbar, das die Oberfläche eines Quaders und eines Würfels aus Rechtecken und Quadraten besteht. Die Summe dieser Flächen bildet den Oberflächeninhalt eines Körpers.
Kinder müssen auch für Körper ohne bekanntes Netz erkennen, aus welchen Flächen die Oberfläche besteht. Hier können mit Sachaufgaben Einheiten und komplizierte Körper abgefragt werden.
\subsection{Beispiele}
\subsection{Fragen}
\end{document}