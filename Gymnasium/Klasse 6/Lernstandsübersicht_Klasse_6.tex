\documentclass{article}
\usepackage[ngerman]{babel}
\usepackage[margin=2cm]{geometry}
\usepackage{amsmath, amssymb, amsfonts}
\usepackage{tikz}
\usepackage{graphicx}
\usepackage{fancyhdr}

\usepackage{tcolorbox} %Umgebung mit farbigem Hintergrund

\graphicspath{{../../}}
%Seitenstil modifizieren
\pagestyle{fancy}% eigenen Seitestil aktivieren}
\fancyhf{}% Alle Felder loeschen

\fancyhead[L]{
\begin{tabular}[b]{l}
Lernstandsübersicht\\
Mathematik\\
Klasse 6
\end{tabular}}
\fancyhead[R]{
\includegraphics[height=3\baselineskip]{Wissensklub-Logo.png}
}
\addtolength{\headheight}{2\baselineskip}
\addtolength{\headheight}{0.61pt}
\fancyhead[C]{2024}

\begin{document}
\begin{titlepage}
    \begin{center}
        \vspace*{1cm}
            
        \Huge
        \textbf{Lernstandsübersicht}\\            
        \vspace{0.5cm}
        \LARGE
        Mathematik
            
        \vspace{1.5cm}
            
        \textbf{Wissensklub GmbH}
            
        \vfill
            
        Klasse 6\\
        \textit{Gymnasium} Nordrhein-Westfalen
            
        \vspace{0.8cm}
            
        \includegraphics[width=0.5\textwidth]{Wissensklub-Logo.png}
            
        \Large
        2024          
    \end{center}
\end{titlepage}
\section{Brüche}
Es werden Brüche zunächst als Darstellung von Anteilen eingeführt.
Anteile können mit Brüchen verglichen werden und in Prozent angegeben werden.
Anschließend werden Brüche auch als Zahlen verstanden. 
\textbf{Für dieses Thema sind aus der 5. Klasse die Themen \textit{Schriftliche Multiplikation und Division}, \textit{Teilbarkeit} und \textit{Zahlen Ordnen} Vorraussetzung.}
\subsection{Inhalte}
\subsubsection*{Bruch und Anteil}
\textit{In wie viele Teile kann ich einen Kuchen schneiden? Wie kann ich eine Schokoladentafeln brechen?} \\
Der Bruch wird eingeführt zur Beschreibung von Anteilen an einem Ganzen.
Der \textit{Zähler} gibt an wie viele Teile man hat bzw. zählt, der \textit{Nenner} gibt an wie viele Teile in einem Ganzen sind.
Anhand von Bildern kann man Anteile feststellen und diese als Bruch aufschreiben.
Durch bildliches Aufteilen kann man mit Brüchen Anteile einer Zahl mit und ohne Einheit berechnen.
Wenn ein Anteil bekannt ist, können die Kinder damit das Ganze berechnen.
\subsubsection*{Kürzen und Erweitern}
\textit{Bei welchen Aufteilungen bleibt eine Gruppengröße gleich? }\\
Es gibt verschieden aussehende Brüche die jedoch den gleichen Anteil beschreiben. Das liegt daran, das das Verhältnis von Nenner und Zähler bei diesen Brüchen gleich ist.
Wenn man Nenner mit der gleichen Zahl multipliziert bzw. dividiert, bleibt der Anteil gleich. Man nennt dies \textit{Erweitern} bzw. \textit{Kürzen}.
Es sollen anhand von Figuren festgestellt werden, welche Aufteilungen die gleichen Anteile beschreiben und welche Brüche dies sind. Dazu soll auch das Kürzen und Erweitern nur mit dem Bruch ausgeführt werden können.
Da man nun Brüche als Verhältnis verstehen kann, sollen für andere Aufteilungen entsprechende Anteile aus nicht vollständigen Informationen hergeleitet werden können.
\subsubsection*{Brüche vergleichen}
\textit{Wer hat im Durchschnitt bei 20 Würfelwürfen das höhere Ergebnis?}\\
Wenn Brüche als Bilder, z.B. durch Flächeninhalte, veranschaulicht werden, ist es gut zu sehen welche Anteile größer sind als andere.
Kinder sollen durch das Erweitern und Kürzen von Brüchen diese auf den gleichen Nenner bringen (kleinste gemeinsame Vielfache) und diese anhand ihres Zählers vergleichen.
Zusätzlich sollen Kinder auch bei gleichem Zähler und unterschiedlichem Nenner, z.B: $\frac{4}{5}, \frac{4}{6}, \frac{4}{7}$ diese Brüche der Größe ordnen können.
\subsubsection*{Prozente}
\textit{Um wie viel Euro sind die Waren im Angebot günstiger?}\\
Prozente sind Anteile, also Brüche, die in Hundertstel Teilen gemessen werden. Hier gilt die Schreibweise $1\% = \frac{1}{100}$.
Es sollen die Vorteile von dem Rechnen mit Prozenten verstanden werden. 
Weiterhin sollen Kinder andere Brüche in Prozent und in Prozentschreibweise umformen können. Auch Zahlen in Prozentschreibweise sollen in Brüche umgeformt und gekürzt werden.
Prozente eignen sich insbesondere für Sachaufgaben mit oder ohne Einheiten wie z.B. bei Angeboten oder Rezepten.
\subsubsection*{Brüche als Quotienten}
\textit{Was machen wir wenn wir eine Waffel mit fünf Herzen an zwei Kinder verteilen wollen?}\\
Brüche erhalten eine neue Interpretation über dem Verständnis als Anteil hinaus: Ein Bruch ist das Ergebnis einer Division von Zähler und Nenner.
Kinder sollen das Ergebnis von Divisionen als Bruch und/oder mit geeigneten Schaubildern darstellen können .
Hier werden auch Brücher mit größerem Zähler als Nenner sinnvoll eingeführt.
\subsubsection*{Brüche auf dem Zahlenstrahl}
\textit{Wo passen Brüche auf den Zahlenstrahl mit den ganzen Zahlen?}\\
Mithilfe von Brüchen können wir die Räume zwischen zwei ganzen Zahlen weiter unterteilen.
Dadurch können wir Brüchen mit einem größeren Zähler als Nenner einen gemischten Bruch auf dem Zahlenstrahl zuordnen.
Hier muss die Strecke sinnvoll geteilt werden um Brüche richtig einzutragen.
\subsection{Beispiele}
\begin{tcolorbox}[colback=gray!5!white,colframe=gray!25!black]
Test
\end{tcolorbox}
\subsection{Fragen}
\begin{tcolorbox}[colback=blue!5!white,colframe=blue!25!black]
Test
\end{tcolorbox}
\newpage
\section{Dezimalzahlen}
In diesem Kapitel werden Dezimalzahlen über Brüche eingeführt. Diese werden im Kontext von Einheiten genauer betrachtet.
\textbf{Wichtig für dieses Kapitel sind das gesamte Kapitel 1 \textit{Brüche} und aus der 5. Klasse \textit{Schriftliches Rechnen} (insbesondere Division) und \textit{Rechnen mit Einheiten}}.
\subsection{Inhalte}
\subsubsection*{Dezimalschreibweise}
\textit{Wie kann ich einen Bruch als Dezimalzahl schreiben?}\\
Die Brüche in denen im Nenner ein Vielfaches von Hundert steht, also $10,100,1000,\ldots$, lassen sich besonders leicht in Dezimalschreibweise umformen.
Hier lernen Kinder wie sie anhand des Nenners das Komma für die entsprechende Dezimalzahl setzen.
Die Stellenwerttafel wird in die Richtung Zehntel, Hundertstel, \ldots erweitert. Die Kinder sollen Brüche in Dezimalzahlen umformen können und auch in Prozentschreibweise notieren.
\subsubsection*{Dezimalzahlen vergleichen und runden}
\textit{Wie kannn ich bei Zahlen in Dezimalschreibweise eine Ordnung finden?}\\
Genauso wie beim Runden von ganzen Zahlen werden Kommazahlen gerundet. Hier sollen Kinder für eine gegebene Rundungsstelle Zahlen auf- bzw. abrunden können.
Das Vergleichen von Dezimalzahlen ist schnell gelöst: Die Kinder müssen die Ziffern einen Dezimalzahl ihren Stellenwert (Zehner, Einer, Zehntel, \ldots) zuordnen können und dann anhand der Übereinstimmung entscheiden, welche Zahl größer bzw. kleiner ist.
\subsubsection*{Abbrechende und periodische Dezimalzahlen}
\textit{Was mache ich wenn bei der schriftlichen Division sich die Schritte wiederholen?}\\
Viele Brüche brechen nicht ab, d.h. das eine Zahl unendlich viele sich wiederholende Stellen in Dezimalschreibweise hat.
Kinder müssen hier die schriftliche Division gut beherrschen, und dann selber feststellen wann der Algorithmus in einem Kreislauf gefangen ist.
Entsprechend schreiben wir eine Dezimalzahl in perioscher Schreibweise, z.b. $0,2\overline{63}$.
Das Vergleichen zweier periodischer Dezimalzahlen nach ihrer Größe wird erprobt.
\subsubsection*{Dezimalschreibweise bei Größen}
\textit{Wie rechne ich Einheiten um, wenn die Zahl nicht vollständig in die nächstgrößere passt?}\\
Nachdem die Kinder gelernt haben, welchen Wert die Stellen im Dezimalsystem haben, und das eine Multiplikation bzw. Division einer Kommaverschiebung nach rechts bzw. links bedeutet, wird dies kombiniert mit dem Umrechnen von Einheiten.
Je nach Umrechnungsfaktor wird das Komma eine unterschiedliche Anzahl an Stellen verschoben.
Wenn Kinder Größen unterschiedlicher Einheit vergleichen sollen, müssen sie diese geschickt umwandeln für einen sinnvollen Vergleich.
\subsection{Beispiele}
\begin{tcolorbox}[colback=gray!5!white,colframe=gray!25!black]
Test
\end{tcolorbox}
\subsection{Fragen}
\begin{tcolorbox}[colback=blue!5!white,colframe=blue!25!black]
Test
\end{tcolorbox}
\newpage
\section{Rechnen mit Brüchen und Dezimalzahlen}
In diesem Kapitel wird das Rechnen mit Brüchen und Dezimalzahlen erprobt. Rechenregeln werden geschickt eingesetzt und die Eigenschaften der jeweiligen Darstellung ausgenutzt.\\
\textbf{Für dieses Kapitel ist ein sehr gutes Verständnis von Thema 1 und 2 nötig! Aus der Klasse 5 sind die Themen \textit{Terme}, \textit{Rechengesetze} und \textit{Schriftliches Rechnen} notwendig.}
\subsection{Inhalte}
\subsubsection*{Brüche addieren und subtrahieren}
\textit{Wie viel sind zwei Viertel Äpfel zusammen?}\\
Brüche mit gleichem Nenner können addiert oder subtrahiert werden, indem der Nenner beibehalten wird und die Zähler verrechnet werden.
Hier müssen Kinder durch das geschickte Erweitern und Kürzen auf den gemeinsamen Nenner bringen, oder ggf. Zahlen in Bruchschreibweise zunächst umformen.
Kinder sollten den ungefähren Wert von Brüchen überschlagen können, um ein Ergebnis einordnen zu können.
\subsubsection*{Dezimalzahlen addieren und subtrahieren}
\textit{Was ist die Gesamtlänge von drei Weitsprüngen?}\\
Das Rechnen mit Dezimalzahlen funktioniert dank der erweiterten Stellenwerttabelle hinter dem Komma genauso wie bei der bekannten schriftlichen Addition.
Alternativ kann ein Zusammenhang mit der Bruchrechnung durch Umformung hergestellt werden.
Durch das Runden kann ein Überschlag gemacht werden.
Bei Termen sollen die Kinder mit dem Kommutativ- und Assoziativgesetz gut umgehen können um nicht unnötig viele Brüche umformen zu müssen oder Minusklammern geschickt auzulösen oder einzusetzen.
\subsubsection*{Brüche vervielfachen und teilen}
Brüche können mit einer natürlichen Zahl multipliziert werden indem der Zähler mit dieser Zahl verrechnet und der Nenner beibehalten wird.
Bei der Division mit einer natürlichen Zahl wird der Nenner mit dieser Zahl multipliziert und der Zähler beibehalten.
Falls der Zähler durch den Divisor teilbar ist, kann dieser auch direkt geteilt und der Nenner beibehalten werden.
\subsubsection*{Brüche Mulitiplizieren und Dividieren}
\textit{Wie berechne ich den Anteil von einem Anteil?}
Wenn Brüche multipliziert werden, werden Zähler mit Zähler sowie Nenner und Nenner seperat multipliziert.
Dies ist die gleiche Art wie ein Anteil einer natürlichen Zahl berechnet wird: Somit können wir auch nun Anteile von Anteilen berechnen.
Kinder sollten geschickt Kürzen können um den Rechenaufwand frühzeitig zu verringern. \\
Die Division mit Brüchen kann man durch eine logische Fortsetzung des Teilens immer kleinerer Zahlen herleiten: Hier merken die Kinder, dass das Teilen mit immer kleineren Zahlen, z.B. $\frac{1}{2}, \frac{1}{4}, \frac{1}{8}, \ldots$ das gleiche wie eine Multiplikation mit $2, 4, 8, \ldots$ ist.
In anderen Worten: Man dividiert einen Bruch indem man mit dem Kehrwert multipliziert.
Kinder sollen mit frühem Kürzen Brüche dividieren können und in Sachaufgaben verwenden.
\subsubsection*{Dezimalzahlen Mulitiplizieren und Dividieren}
Für dieses Thema ist das Konzept der Kommaverschiebung der Schlüssel.
Kinder sollen Dezimalzahlen Mulitiplizieren indem sie zuerst das Komma ignorieren, und anschließend anhand der Summe der Nachkommastellen der Faktoren im Ergebnis das Komma an die richtige Stelle setzen.
Dadurch sollen Kinder auch das Ergebnis weiterer Aufgaben, in denen das Komma an einer anderen Stelle steht aber die Ziffern gleich sind, ohne Rechnung aufschreiben können.
Bei der Division müssen sich die Kinder bewusst werden, das wenn man bei Dividend und Divisor das Komma gleich verschiebt, das Ergebnis gleich bleibt, z.B. $200 : 50 = 20:5 = 2:0.5 = 0.02 : 0.005 = 4$.
Dann verändern die Kinder den Term so, dass der Divisor eine natüürliche Zahl ist und beachten das Komma beim Endergebnis.
Hier sollen auch wieder die Rechengesetze genutzt werden, um schwierige Aufgaben in der bestmöglichen Reihenfolge zu rechnen.
Hier lernen die Kinder das erste Mal das Distributivgesetz und die Konzepte Ausmultiiplizieren und Ausklammern kennen.
\subsection{Beispiele}
\begin{tcolorbox}[colback=gray!5!white,colframe=gray!25!black]
Test
\end{tcolorbox}
\subsection{Fragen}
\begin{tcolorbox}[colback=blue!5!white,colframe=blue!25!black]
Test
\end{tcolorbox}
\newpage
\section{Muster und Figuren}
In diesem Thema werden Kreise, Kreisfiguren und Winkel eingeführt und damit auch das Arbeiten mit dem Zirkel. 
\textbf{Hier sind die vorigen Themen \textit{Koordinatensysteme}, \textit{Achsen- und Punktsymmetrie} und der Umgang mit dem Geodreieck Vorraussetzung!}
\subsection{Inhalte}
\subsubsection*{Negative Zahlen im Koordinatensystem}
Kinder lernen hier das erste Mal negative Zahlen kennen, indem das Koordinatensystem gespiegelt in die bisher bekannte Richtung weitergeführt wird.
Kinder sollen das neue Koordinatensystem bschriften können, Punkte eintragen, ablesen und darstellen können, sowie eine Achsen- und Punktspiegelung durchführen können.
Diese Konzepte sind alle aus der 5. Klasse bekannt, die Schwierigkeit besteht in dem Umgang mit den neuen Zahlen.
\subsubsection*{Verschiebungen}
\textit{Wie setzt sich ein Tapetenmuster fort?}\\
Bei einer Verschiebung  werden alle Punkte gleich weit in die gleiche Richtung verschoben. Dabei können einzelne Punkte aber auch ganze Figuren verschoben werden.
Kinder müssen diese Verschiebung im Koordinatensystem übertragen und auch mit Spiegelungen kombinieren können.
\subsubsection*{Kreise und Kreisfiguren}
Kreise sind Formen, in denen jeder Randpunkt den gleichen Abstand zur MItte hat. Ein Kreis hat einen Mittelpunkt, der Abstasnd zum Rand heißt Radius.
Kinder lernen mit dem Zirkel Kreise zu zeichnen, und Kreisfiguren wie z.B. sich wiederholende Halbkreise im Rechteck zu zeichnen.
Kreise werden nach Beschreibung von Mittelpunkt und Radisum im Koordinatensystem eingezeichnet und elementare Eigenschaften beobachtet.
\subsubsection*{Winkel}
\textit{Warum kann der Torwart den Ball besser von der Seit als frontal von vorne abwehren?}\\
Ein Winkel beschreibt die Größe der Öffnung die von zwei Geraden, den sogenannten Schenkeln aufgespannt wird. Die Kinder lernen griechische Buchstaben kennen um Winkel zu beschreiben.
Die Kinder lernen verschiedene Winkelarten kennen, wie  z.B. spitze, rechte und stumpfe Winkel. Es wird gelernt Winkel in Grad zu messen, als einer Einteilung des Kreises in 360 gleich große Teile.
Es werden Winkel mittel der Schenkel oder drei Punkten, dem Scheitel und zwei Punkten auf den Schenkeln, bezeichnet.
Es wird erforscht, welche Arten von Winkel in Vielecken vorkommen können.
\subsubsection*{Winkel Messen und Zeichnen}
Mit dem Geodreieck sollen Winkel gemessen und gezeichnet werden. Das Geodreick muss sauber im Scheitel angelegt werden, dann muss der Winkel anhand der richtigen Skala abgelesen werden.
Es werden Techniken zum Ablesen überstumpfer Winkel erarbeitet. Winkel müssen abgeschätzt werden können anhand einer Einteilung des Kreises in vier rechte Winkel, z.B. $180° < \alpha < 270°$.
Winkel müssen ohne Rechnung abgelesen werden können, z.B. wenn zwei oder mehrere Winkel zusammen einen Halbkreis oder ganzen Kreis ergeben.
\subsubsection*{Drehungen}
Figuren werden von einem Drehzentrum aus mit einem Drehwinkel im Kreis rotiert. Vom Drehzentrum aus müssen Eckpunkte einer Figur mit dem Zirkel an die neue Stelle übertragen werden, anschließend kann die Figur verbunden werden.
Es sollen von bestehenden Drehungen aus der Drehwinkel abgelesen werden können mit dem Geodreick. Im Koordinatensystem werden Punkte vorgegeben, diese sollen zu einer bestimmten Figur verbunden werden und anschließend rotiert werden.
\subsection{Beispiele}
\begin{tcolorbox}[colback=gray!5!white,colframe=gray!25!black]
Test
\end{tcolorbox}
\subsection{Fragen}
\begin{tcolorbox}[colback=blue!5!white,colframe=blue!25!black]
Test
\end{tcolorbox}
\newpage
\section{Daten}
Das bisherige Wissen wird um Kreisdiagramme und Boxplots erweitert, sowie relativen und absoluten Häufigkeiten, arithmetischem Mittel, Median und Quantielen. Es wird das Grundwissen von Versuchsaufbau und -auswertung vermittelt.
\textbf{Vorraussetzung ist aus der 5. Klasse \textit{Zählen und Darstellen}, \textit{Brüche}, \textit{Dezimalzahlen} und  \textit{Kreise und Kreisfiguren}.}
\subsection{Inhalte}
\subsubsection*{Relative Häufigkeiten und Diagramme}
\textit{Welche Sportgruppe unterschiedlicher Größe hat einen größeren Anteil an Leistungssportlern?}
Die relative Häufigkeit ist das Verhältnis von Anzahl, also der absoluten Häufigkeit, und der Gesamtzahl.
Relative Häufigkeiten sind von Natur aus Brüche und werden häufig in Prozent festgehalten.
Relative Häufigkeiten und Prozente müssen anhand von einer Statistik, also gesammelten Werten, ausgerechnet werden, umgeformt werden können in Prozent, und dann in Säulen- Blaken und Kreisdiagramme eingetragen werden.
Kreisdiagramme messen relative Häufigkeiten anhand ihres Anteils des Vollwinkels $360°$.   
Kinder sollen Kreisdiagramme ablesen können, selber einzeichnen und Anteile abschätzen können.
\subsubsection*{Median und arithmetisches Mittel}
Der Median ist der mittlere Wert einer sortierten Liste. Das arithmetische Mittel oder Durchschnitt, die Summe aller Werte geteilt durch die Gesamtzahl.
Es sollen beide Konzepte unterschieden und berechnet werden können. Vor- und Nachteile dieser Begriffe werden besprochen, insbesondere ihre Aussagekraft.
Es sollen Rechenvorteile beim Berechnen vom arithmetischem MIttel aus Häufigkeitstabellen genutzt werden.
\subsubsection*{Boxplots}
Boxplots enthalten viele Informationen über die Verteilung von Werten einer Statistik, insbesondere über die Lage von Werten nach Quartilen unterteilt.
An einem Boxplot sollen die Begriffe Minimum, Maximum, oberes- und unteres Quartil, Median, Spannweite und untere- und obere Antenne identifiziert werden können.
Anhand von Tabellen oder Wertelisten sollen Boxplots erstellt werden.
Verschiedene Statistiken sollen anhand eines Boxplots verglichen werden.
\subsection{Beispiele}
\begin{tcolorbox}[colback=gray!5!white,colframe=gray!25!black]
Test
\end{tcolorbox}
\subsection{Fragen}
\begin{tcolorbox}[colback=blue!5!white,colframe=blue!25!black]
Test
\end{tcolorbox}
\newpage
\section{Beziehungen zwischen Zahlen}
In diesem Thema wird die Struktur des Rechnens weiter analysiert und Regelmäßigkeiten gefunden. Daraus wird der Dreisatz erlernt und Rechenausdrücke mit Variablen beschrieben.
\textbf{Vorraussetzung ist das\textit{Rechnen mit Dezimalzahlen und Brüchen}, \textit{Koordinatensysteme}, \textit{Terme} und \textit{Rechenregeln}}.
\subsubsection*{Zahlenfolgen und Terme}
Für eine gegebene Folge von Zahlen, z.B. $2,4,6,8,\ldots$ sollen die Nachfolger anhand der Struktur gefunden werden. 
Über die Regelmäßigkeit wird ein Term hergeleitet, der für eine beliebige Stelle direkt die Zahl in der Folge berechnet. Dies ist ein Term mit einer Variablen.
Es sollen Terme aufgestellt werden können und aus einem Sachkontext abgeleitet werden. Es werden exemplarisch Terme für einen bestimmten Wert der Variable ausgerechnet.
\subsubsection*{Dreisatz}
Der Dreisatz wird benutzt um Werte abhängiger Zahlenpaare mit einer Zwischenrechnung auszurechnen. Das ist z.B. für Preise von Lebensmitteln (\textit{je-mehr-desto-mehr}) oder die Aufteilung von Wasser auf Gläser (\textit{je-mehr-desto-weniger}) nützlich.
Hier werden Sachaufgaben intensiv genutzt, damit ein Zusammenhang zunächst festgestellt werden muss. Anschließend soll der Dreisatz angewendet werden können.
\subsubsection*{Abhängigkeiten grafisch darstellen}
Abhängigkeiten wie im Dreisatz können als Zahlenpaar in einem Koordinatensystem eingetragen werden. Dieser Zusammenhang kann linear sein, also wenn alle Punkte sich durch eine gerade Linie verbinden lassen.
Anhand eines Graphen in einem Koordinatensystem soll ein Zusammenhang abgelesen werden. Aus einer Wertetabelle soll ein Diagramm erstellt werden können, zunächst durch das Einzeichnen von Punkten und dann angemessenes Verbindung durch eine Linie. 
\end{document}