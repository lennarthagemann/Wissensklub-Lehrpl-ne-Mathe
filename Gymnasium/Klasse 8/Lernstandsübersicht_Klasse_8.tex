\documentclass{article}
\usepackage[ngerman]{babel}
\usepackage[margin=2cm]{geometry}
\usepackage{amsmath, amssymb, amsfonts}
\usepackage{tikz}
\usepackage{graphicx}
\usepackage{fancyhdr}

\usepackage{tcolorbox} %Umgebung mit farbigem Hintergrund

\graphicspath{{../../}}
%Seitenstil modifizieren
\pagestyle{fancy}% eigenen Seitestil aktivieren}
\fancyhf{}% Alle Felder loeschen

\fancyhead[L]{
\begin{tabular}[b]{l}
Lernstandsübersicht\\
Mathematik\\
Klasse 7
\end{tabular}}
\fancyhead[R]{
\includegraphics[height=3\baselineskip]{Wissensklub-Logo.png}
}
\addtolength{\headheight}{2\baselineskip}
\addtolength{\headheight}{0.61pt}
\fancyhead[C]{2024}

\begin{document}
\begin{titlepage}
    \begin{center}
        \vspace*{1cm}
            
        \Huge
        \textbf{Lernstandsübersicht}\\            
        \vspace{0.5cm}
        \LARGE
        Mathematik
            
        \vspace{1.5cm}
            
        \textbf{Wissensklub GmbH}
            
        \vfill
            
        Klasse 7\\
        \textit{Gymnasium} Nordrhein-Westfalen
            
        \vspace{0.8cm}
            
        \includegraphics[width=0.5\textwidth]{Wissensklub-Logo.png}
            
        \Large
        2024          
    \end{center}
\end{titlepage}

\section{Lineare Funktionen}
In diesem Kapitel wird das Wissen zu proportionalen Zuordnungen mit Funktionen verknüpft.\\
\textbf{Voraussetzungen: }
\begin{itemize}
    \item Zuordnungen: grafisch und algebraisch darstellen, Zuordnungstypen erkennen (Klasse 7)
    \item Terme und Gleichungen: Terme aufstellen, Werte eines Terms berechnen, Äquivalenzumformungen (Klasse 7)
\end{itemize}
\subsection{Inhalte}
\subsubsection*{Funktionen}
\subsubsection*{Funktionen mit der Gleichung y = mx + b}
\subsubsection*{Funktionsgleichungen bestimmen}
\subsubsection*{Nullstellen und Schnittpunkte}
\newpage
\section{Terme mit mehreren Variable}
\subsection{Inhalte}
\subsubsection*{Multiplizieren von Summen}
\subsubsection*{Binomische Formeln}
\newpage
\section{Flächen}
\subsection{Inhalte}
\subsubsection*{Parallelogramme}
\subsubsection*{Dreiecke}
\subsubsection*{Zusammengesetzte Figuren}
\newpage
\section{Lineare Gleichungssysteme}
\subsubsection*{Lineare Gleichungen mit zwei Variablen}
\subsubsection*{Gleichsetzungs- und Einsetzungsverfahren}
\subsubsection*{Additionsverfahren}
\subsubsection*{Probleme mit Gleichungssystemen}
\newpage

\section{Kreise und Dreiecke}
\subsection{Inhalte}
\subsubsection*{Der Satz des Thales}
\subsubsection*{Mittelsenkrechte und Außenkreis}
\subsubsection*{Winkelhalbierende und Inkreis}
\subsubsection*{Schwerpunkt eines Dreiecks}
\end{document}