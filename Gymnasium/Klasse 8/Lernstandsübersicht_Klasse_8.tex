\documentclass{article}
\usepackage[ngerman]{babel}
\usepackage[margin=2cm]{geometry}
\usepackage{amsmath, amssymb, amsfonts}
\usepackage{tikz}
\usepackage{graphicx}
\usepackage{fancyhdr}

\usepackage{tcolorbox} %Umgebung mit farbigem Hintergrund

\graphicspath{{../../}}
%Seitenstil modifizieren
\pagestyle{fancy}% eigenen Seitestil aktivieren}
\fancyhf{}% Alle Felder loeschen

\fancyhead[L]{
\begin{tabular}[b]{l}
Lernstandsübersicht\\
Mathematik\\
Klasse 8
\end{tabular}}
\fancyhead[R]{
\includegraphics[height=3\baselineskip]{Wissensklub-Logo.png}
}
\addtolength{\headheight}{2\baselineskip}
\addtolength{\headheight}{0.61pt}
\fancyhead[C]{2024}

\begin{document}
\begin{titlepage}
    \begin{center}
        \vspace*{1cm}
            
        \Huge
        \textbf{Lernstandsübersicht}\\            
        \vspace{0.5cm}
        \LARGE
        Mathematik
            
        \vspace{1.5cm}
            
        \textbf{Wissensklub GmbH}
            
        \vfill
            
        Klasse 8\\
        \textit{Gymnasium} Nordrhein-Westfalen
            
        \vspace{0.8cm}
            
        \includegraphics[width=0.5\textwidth]{Wissensklub-Logo.png}
            
        \Large
        2024          
    \end{center}
\end{titlepage}

\section{Lineare Funktionen}
In diesem Kapitel wird das Wissen zu proportionalen Zuordnungen mit Funktionen verknüpft.\\
\textbf{Voraussetzungen: }
\begin{itemize}
    \item \textbf{Zuordnungen}: grafisch und algebraisch darstellen, Zuordnungstypen erkennen (Klasse 7)
    \item \textbf{Terme und Gleichungen}: Terme aufstellen, Werte eines Terms berechnen, Äquivalenzumformungen (Klasse 7)
\end{itemize}
\subsection{Inhalte}
\subsubsection*{Funktionen}
Funktionen sind Zuordnungen, bei denen der Ursprungswert ein eindeutiger (Funktions-)wert zugeordnet werden kann.
Es werden die Begriffe Stelle, Funktionswert, Punkt, Funktionsterm und -graph eingeführt.
Es sollen Funktionsgraphen gelesen werden können, Punkte auf einem Graphen bestimmt und im Sachkontext eingeordnet werden können. 
Zu entsprechnenden Wertetabellen werden Funktionsgraphen und -terme aufgestellt. 
Zu gegebenen Graphen soll entschieden werden ob dieser zu einer Funktion gehören kann oder nicht.
\subsubsection*{Funktionen mit der Gleichung $f(x) = m \cdot x$}
Ein wichtiger und häufig auftretener Zuordnungstyp ist ein linearer, und hier werden lineare Funktionen eingeführt.
Zunächst wird für $f(x) = m \cdot x$ die sogenannte Ursprungsgerade als Funktionsgraph eingeführt. $m$ stellt die Steigung dar, und mithilfe des Steigungsdreieck wird die Steigung anhand des Funktionsgraph abgelesen.
Somit soll ein Graph zu einer Funktionsgleichung gezeichnet werden sowie eine Funktionsgleichung zu einem gegebenen Graphen aufgestellt werden.
Zu gegebenen Punkten soll überprüft werden, ob dieser auf dem Funktionsgraph liegt, grafisch wie rechnerisch.
\subsubsection*{Funktionen mit der Gleichung $f(x) = m \dot x + b$}
Eine Funktion mit der Gleichung $f(x) = m \cdot x + b$ heißt lineare Funktion. Der Graph entspricht einer Ursprungsgerade mit Steigung $m$ und einer Verschiebung $b$ in der y-Achse, auch y-Achsenabschnitt genannt.
Es sollen Graphen zu bestimmten Gleichungen gezeichnet werden, die Funktionsgleichung abgelesen werden können und die Punktprobe durchgeführt werden.
Lineare Funktionen sollen im Sachzusammenhang aufgestellt und abgelesen werden können.
\subsubsection*{Funktionsgleichungen bestimmen}
Es werden Verfahren erarbeitet, um die Funktionsgleichung einer linearer Funktionen rechnereisch zu ermitteln. Hierfür wird die Methode mit zwei verschiedenen bekannten Punkten eingeübt um Steigung und y-Achsenabschnitt sukzessive zu ermitteln.
Es soll dieses Verfahren genutzt werden um Funktionsgleichungen zu bestimmen, rein rechnerisch oder mit zusätzlicher Sachkomponente.
\subsubsection*{Nullstellen und Schnittpunkte}
In vielen Kontexten ist es wichtig zu wissen, wann eine Funktion den Wert Null erreicht. Demnach ist die Frage die sich stellt: Für welche $x$-Wert erhalte ich $f(x) = 0$?
Dazu gibt es einen grafischen und rechnerischen Ansatz um die Nullstelle zu bestimmen. Dieses Wissen wird übertragen auf die Fragestellung, wann eine Funktion einen bestimmt Wert erreicht, z.B. wann gilt $f(x) = 5$?
Im Sachkontext soll interpretiert werden, was es bedeutet wenn zwei Funktionen den gleichen Wert haben, oder was eine Nullstelle bedeutet.
\newpage
\section{Terme mit mehreren Variable}
Terme werden erweitert und verschiedene Variablen werden zugelassen. Es wird gelernt warum mehrere Variablen nützlich sein können um komplexe Zusammenhänge zu beschreiben und Probleme zu lösen. Dazu wird das vereinfachen und berechnen von Termen mit mehreren Variablen eingeübt.
Die Binomischen Formeln werden hergeleitet und als \textit{Rechenstrategie} eingesetzt.\\
\textbf{Voraussetzungen: }
\begin{itemize}
    \item \textbf{Terme}: Terme mit rationalen Zahlen und mit einer Variablen berechnen und vereinfachen
    \item \textbf{Terme und Gleichungen}: Terme aufstellen, Gleichungen mit Äquivalenzumformungen lösen, Sachaufgaben mittels Termen und Gleichungen lösen
    \item \textbf{Potenzen}: Potenzschreibweise, Produkte zu Potenzen zusammenfassen
    \item \textbf{Rechengesetze}: Insbesondere Potenz- und Distributivgesetz
\end{itemize}
\subsection{Inhalte}
\subsubsection*{Terme mit mehreren Variablen}
Um z.B. Den Oberflächeninhalt eines Quaders als Formel aufzuschreiben, benötigen wir dafür drei Variablen.
Wenn wir einen Term mit mehreren Variablen haben, können wir die Summanden mit denselben Variablen und Potenzen zusammenfassen.
Wir fassen Terme mit mehreren Variablen zusammen, indem wir Produkte als Potenzen aufschreiben und passende Summanden zusammenfassen.
Anhand von zusammengesetzen Flächen und Formen können wir Formeln z.B. für die Fläche mit mehreren Variablen in einem Term aufstellen und zusammenfassen.
Wir können verschiedene Terme durch Aquivalenzumformungen auf Gleichwertigkeit überprüfen.
\subsubsection*{Multiplizieren von Summen}
Das Distributivgesetz ist bereits bekannt um einen Wert mit einer Klammer zu multiplizieren. 
Wenn wir zwei Klammern mit je zwei Summanden multiplizieren, benutzen wir das Distributivgesetz zweimal.
Dies lässt sich grafisch herleiten mit einem Rechteck dessen Seiten in je zwei verschiedene Längen unterteilt werden.
Es sollen Terme mit Produkte von Klammen ausmultipliziert und anschließend vereinfacht werden können.
\subsubsection*{Binomische Formeln}
Die Binomische Formlen sind einfache Rechenregeln für die Produkte von Klammern bestimmter Form.
Die drei binomischen Formeln werden hergeleitet mit der bekannten Herangehensweise für das Multiplizieren von Summen.
Danach sollen die binomischen Formeln angewendet werden im Kontext von Termen. Diese sollen auch \textit{rückwärts} eingesetzt werden, um damit einen Term zu faktorisieren.
Dann sollen die binomischen Formeln als Rechentrick genutzt werden, um z.B. große quadrateische Zahlen auszurechnen wie $41^2$.
\newpage
\section{Flächen}
Hier wird sich mit neuen Flächen wie z.B. von Parallelogrammen, Dreiecken und Trapezen und zusammengesetzen Figuren beschäftigt. 
Dabei werden neue Flächeninhaltsformeln aus alten Formeln selbsstständig hergeleitet.
\textbf{Voraussetzungen: }
\begin{itemize}
    \item \textbf{Einheiten}: Insbesondere Flächeneinheiten, Rechnen und Umformen
    \item \textbf{Flächeninhalte}: Quadrate, Rechtecke und rechwinklige Dreiecke, Zusammengesetzte Figuren dieser Formen
    \item \textbf{Formen}: Eigenschaften von Parallelogrammen und Trapezen
    \item \textbf{Umformungen}: z.B. bei Rechtecken Seitenlänge bei gegebenem Flächeninhalt und Länge anderer Seite bestimmen
\end{itemize}
\subsection{Inhalte}
\subsubsection*{Parallelogramme}
Da Flächeninhalte beim Zerschneiden und erneutem Zsuammensetzen von Formen gleichbleiben, wird so die Flächeninhaltsformel für das Parallelogramm  über ein zusammengesetztes Rechteck hergeleitet.
Es soll Grundseite und Höhe eines Parallelogrammes identifiziert werden. Flächeninhalte von Parallelogrammen wird berechnet, diese können durch Punkte, Zeichnungen oder Seitenlängen gegeben werden. 
Die Flächenformel soll genutzt werden um eine fehlender Grundseite oder Höhe zu berechnen, wenn die anderen Werte bekannt sind.
\subsubsection*{Dreiecke}
Jedes Dreieck kann in ein Rechteck eingebettet werden. Durch den Vergleich mit dem umgebenden Rechteck wird die Formel für den Flächeninhalt des Dreiecks hergeleitet.
Die Aufgabentype sind analog zu denen die für das Parallelogramm gestellt werden.
\subsubsection*{Zusammengesetzte Figuren}
Für das Parallelogramm und Dreieck wurden die Figuren erweitert oder zerlegt um eine Flächeninhaltsformel herzuleiten. Dieses Konzept kann man für weitere Figuren benutzen.
So lässt sich auf zwei Wegen, der Ergänzung oder der Zerlegung, eine Formel für den Flächeninhalt des Trapezes herleiten.
Es sollen Strategien zur Mustererkennung bei zusammengesetzen Figuren erlernt werdern. 
Danach sollen bekannte Formeln genutzt werden, um eine Formel für die zusammengesetze Figur zu bestimmen.
\newpage
\section{Lineare Gleichungssysteme}
Hier betrachten wir lineare Gleichungssysteme mit zwei Gleichungen und zwei Variablen.
Hier werden grafische und rechnerische Lösungsstrategien erarbeitet.\\
\textbf{Voraussetzungen: }
\begin{itemize}
    \item \textbf{Lineare Funktionen}: Graphen einer linearen Funktion zeichnen, Funktionsgleichung anhand des Graphen aufstellen
    \item \textbf{Modellierung}: Lineare Gleichung zu einer Sachaufgabe passend aufstellen
    \item \textbf{Lineare Gleichungen}: Lösung einer Gleichung durch Einsetzen überprüfen, Äquivalenzumformungen zur Lösung einer Gleichung einsetzen
    \item \textbf{Terme}: Summanden in einem Term zusammenfassen
\end{itemize}
\subsubsection*{Lineare Gleichungen mit zwei Variablen}
Sobald wir eine Gleichung mit zwei Variablen haben, gibt es potenziell mehrere Lösungen.
Die Lösungen bestehen aus Zahlenpaaren. Die Lösungen der Gleichung werden als Lösungsmenge bezeichnet.
Man kann Lösung mittels Probieren, Auflösen nach eine Variablen und systematischem Einsetzen oder durch Zeichnen des linearen Funktionsgraphen und Ablesen der Punkte lösen.
Hier werden lineare Gleichungen also rechnerisch und grafisch gelöst. Zahlenpaare sollen als Lösung einer Gleichung eingesetzt und überprüft werden können.
\subsubsection*{Lineare Gleichungssysteme grafisch lösen}
Wenn wir eine weitere lineare Gleichung mit zwei Variablen ergänzen, erhalten wir ein lineares Gleichungssystem mit zwei Gleichungen (LGS). Ein Zahlenpaar kann nun eindeutige Lösung des LGS sein, wenn sie eingesetzt in beiden Gleichungen eine wahre Aussage ergibt.
Beide Gleichungen können durch einen Graphen visualisiert werden. Im Schnittpunkt der Geraden liegt die Lösung des LGS. Falls die Geraden sich nicht schneiden, existiert keine Lösung.
Falls die Graphen aufeinanderliegen, gibt es unendlich viele Lösungen und die Lösungsmenge besteht aus allen Punktpaaren auf diesen Graphen.
Es sollen die Lösung(en) eines LGS, falls diese existieren, grafisch ermittelt werden. Es soll entschieden werden, wie viele Lösungen ein vorliegendes LGS hat.
\subsubsection*{Gleichsetzungs- und Einsetzungsverfahren}
Nicht immer lässt sich die grafische Lösung gut ablesen. Hier kommen rechnerische Verfahren ins Spiel.
Beim Gleichsetzungsverfahren werden die Gleichungen jeweils nach einer Variable aufgelöst, und die anderen Seiten dann gleichgesetzt und aufgelöst. So wird ein Lösungswert bestimmt, der andere kann dann durch Einsetzen und Auflösen der bekannten Lösung genauso bestimmt werden.
Beim Einsetzungsverfahren wird nur eine Gleichung nach einer Variablen aufgelöst, und die andere Seite des Terms für die andere Gleichung in der Variablen eingesetzt.
Danach ist nur noch eine Variable vorhanden und die erste Zahl des Lösungspaar kann ermittelt werden. Diese Lösung wird dann in die Ursprungsgleichung eingesetzt um die ganze Lösung zu erhalten.
Es sollen beide Lösungsverfahren genutzt werden um LGS zu lösen. Dabei sollen Muster erkannt werden, wann sich eines der beiden Verfahren besser zur Lösung eignet.
Mit den Lösungsverfahren soll erkannt werden, wann ein LGS keine Lösung, einer oder unendlich viele Lösungen hat. Es soll ein Ergebnis mittels Probe überprüft werden können.
\subsubsection*{Additionsverfahren}
Das Additionsverfahren führt in vielen Fällen am schnellsten zum Ziel. Hier werden die Gleichungen des LGS miteinander addiert. Wenn Variablen in den Gleichungen Gegenzahlen sind, Dann hat die dabei entsandende Gleichung nur noch eine Variable und kann gelöst werden.
Das Ergebnis wird wie von den anderen Verfahren bekannt in die Ausgangsgleichung eingesetzt. Danach wird das Lösungspaar durch eine Probe auf Richtigkeit überprüft.
Das Additionsverfahren soll zur Lösung eines LGS eingesetzt werden. Mit den bekannten Verfahren soll das jeweils am besten geeignete zur Lösung ausgewählt werden. 
Fehler bei der Benutzung der Verfahren sollen selbstständig durch Nachrechnen gefunden und korrigiert werden können.
\subsubsection*{Probleme mit Gleichungssystemen modellieren und lösen}
Sachaufgaben sollen mit LGS modelliert werden. Dazu soll die Aufgabe in seine Teilprobleme zerlegt werden.
\begin{enumerate}
    \item Was ist gesucht, gegeben und wichtig?
    \item Gesuchte Größen mit Variablen benennen, Terme aufstellen
    \item Gleichungen aufstellen und mit I und II bezeichnen
    \item Rechenweg planen: Welches Lösungsverfahren benutze ich?
    \item Rechenweg durchführen
    \item Kontrolle (Probe) und Antwort formulieren
\end{enumerate}
\newpage
\section{Kreise und Dreiecke}
Kreise und Dreiecke werden mit dem Satz des Thales, In- und Umkreisen in Verbindung gesetzt.\\
\textbf{Voraussetzungen: }
\begin{itemize}
    \item \textbf{Dreiecke}: Eigenschaften besonderer Dreiecke
    \item \textbf{Winkel}: Innenwinkelsätze von Dreiecken, Vierecken und Kreisen
    \item \textbf{Dreieckskonstruktionen}: Zeichnen von Dreiecken mit Zirkel und Geodreieck
    \item \textbf{Kongruenzsätze}: Mit den Kongruenzsätzen Dreiecke auf Deckungsgleichheit überprüfen
\end{itemize}
\subsection{Inhalte}
\subsubsection*{Der Satz des Thales}
Mit dem Satz des Thales können rechtwinklige Dreiecke konstruiert werden und fehlende Winkel nachgerechnet werden. Es kann umgekehrt ohne Rechnung entschieden werden, ob ein Dreieck rechtwinklig ist oder nicht.
\subsubsection*{Mittelsenkrechte und Umkreis}
Eine Mittelsenkrechte zu einer Strecke ist eine solche, die durch den Mittelpunkt der Strecke geht und senkrecht zu ihr verläuft.
Zu jeder Seite im Dreieck kann eine Mittelsenkrechte konstruiert werden. Diese treffen sich in einem Punkt, welcher auch Umkreismittelpunkt genannt wird.
In diesem Punkt kann der Umkreis gezeichnet werden, der die Ecken des Dreiecks berührt. Somit hat dieser Punkt den gleichen Abstand zu den Ecken des Dreiecks.
Die Mittelsenkrechte soll zu einer gegebenen Strecke mit Zirkel und Lineal konstruiert werden. Für das Dreieck soll nach diesem Verfahren der Umkreismittelpunkt ermittelt werden und der Umkreis eingezeichnet werden können.
\subsubsection*{Winkelhalbierende und Inkreis}
Eine Winkelhalbierende ist eine Halbgerade, die aus den Ecken des Dreiecks entspringt und den eingeschlossenen Winkel der Schenkel halbiert. Die Winkelhalbierende hat also auf jedem Punkt den gleichen Abstand zu beiden Schenkeln.
Im Schnittpunkt der drei Winkelhalbierenden liegt der Inkreismittelpunkt. In diesem Punkt kann der Inkreis eingezeichnet werden, der alle drei Seiten des Dreiecks tangiert.
Mit dem Zirkel soll die Winkelhalbierende konstruiert und eingezeichnet werden. anschließend soll damit der Inkreismittelpunkt ermittelt und der Inkreis eingezeichnet werden.
\subsubsection*{Schwerpunkt eines Dreiecks}
Die Seitenhalbierende des Dreiecks ist die Strecke zwischen einer Ecke und dem gegenüberliegendem Mittelpunkt der Seite. Im Schnittpunkt der drei Seitenhalbierende liegt der Schwerpunkt des Dreiecks.
Die Seitenhalbierenden und der Schwerpunkt eines Dreiecks soll zeichnerisch ermittelt werden.
\end{document}