\documentclass{article}
\usepackage[ngerman]{babel}
\usepackage[margin=2cm]{geometry}
\usepackage{amsmath, amssymb, amsfonts}
\usepackage{tikz}
\usepackage{graphicx}
\usepackage{fancyhdr}

\usepackage{tcolorbox} %Umgebung mit farbigem Hintergrund

\graphicspath{{../../}}
%Seitenstil modifizieren
\pagestyle{fancy}% eigenen Seitestil aktivieren}
\fancyhf{}% Alle Felder loeschen

\fancyhead[L]{
\begin{tabular}[b]{l}
Lernstandsübersicht\\
Mathematik\\
Klasse 10
\end{tabular}}
\fancyhead[R]{
\includegraphics[height=3\baselineskip]{Wissensklub-Logo.png}
}
\addtolength{\headheight}{2\baselineskip}
\addtolength{\headheight}{0.61pt}
\fancyhead[C]{2024}

\begin{document}
\begin{titlepage}
    \begin{center}
        \vspace*{1cm}
            
        \Huge
        \textbf{Lernstandsübersicht}\\            
        \vspace{0.5cm}
        \LARGE
        Mathematik
            
        \vspace{1.5cm}
            
        \textbf{Wissensklub GmbH}
            
        \vfill
            
        Klasse 10\\
        \textit{Gymnasium} Nordrhein-Westfalen
            
        \vspace{0.8cm}
            
        \includegraphics[width=0.5\textwidth]{Wissensklub-Logo.png}
            
        \Large
        2024          
    \end{center}
\end{titlepage}

\section{Daten und Wahrscheinlichkeit}
\subsection{Inhalte}
\subsubsection*{Statistiken verstehen und beurteilen}
\subsubsection*{Vierfeldertafel - mit Anteilen argumentieren}
\subsubsection*{Bedingte Wahrscheinlichkeiten}
\subsubsection*{Stochastische Unabhängigkeit}
\newpage
\section{Quadratische Gleichungen}
\subsection{Inhalte}
\subsubsection*{Darstellungsformen quadratischer Funktionen}
\subsubsection*{Quadratische Funktionen grafisch lösen}
\subsubsection*{Lösen einfacher quadratischer Gleichungen}
\subsubsection*{Linearfaktorzerlegung}
\subsubsection*{Lösungsformel für quadratische Gleichungen}
\subsubsection*{Modellierung mit quadratischen Gleichungen}
\newpage
\section{Ähnlichkeit}
\subsection{Inhalte}
\subsubsection*{Zentrische Streckung}
\subsubsection*{Ähnlichkeit}
\subsubsection*{Strahlensätze}
\newpage
\section{Exponentielles Wachstum}
\subsection{Inhalte}
\subsubsection*{Exponentielles Wachstum}
\subsubsection*{Exponentialfunktion}
\subsubsection*{Exponentialgleichungen und Logarithmen}
\subsubsection*{Wachstumsprozesse modellieren}
\newpage

\section{Trigonometrie}
\subsection{Inhalte}
\subsubsection*{Sinus und Kosinus}
\subsubsection*{Tangens}
\subsubsection*{Problemlösen mit rechtwinkligen Dreiecken}
\subsubsection*{Der Kosinussatz}
\subsubsection*{Sinus und Kosinus am Einheitskreis}
\subsubsection*{Die Sinusfunktion}
\subsubsection*{Transformationen der Sinusfunktion}
\subsubsection*{Beschreibung periodischer Vorgänge}
\end{document}