\documentclass{article}
\usepackage[ngerman]{babel}
\usepackage[margin=2cm]{geometry}
\usepackage{amsmath, amssymb, amsfonts}
\usepackage{tikz}
\usepackage{graphicx}
\usepackage{fancyhdr}

\usepackage{tcolorbox} %Umgebung mit farbigem Hintergrund

\graphicspath{{../../}}
%Seitenstil modifizieren
\pagestyle{fancy}% eigenen Seitestil aktivieren}
\fancyhf{}% Alle Felder loeschen

\fancyhead[L]{
\begin{tabular}[b]{l}
Lernstandsübersicht\\
Mathematik\\
Klasse 10
\end{tabular}}
\fancyhead[R]{
\includegraphics[height=3\baselineskip]{Wissensklub-Logo.png}
}
\addtolength{\headheight}{2\baselineskip}
\addtolength{\headheight}{0.61pt}
\fancyhead[C]{2024}

\begin{document}
\begin{titlepage}
    \begin{center}
        \vspace*{1cm}
            
        \Huge
        \textbf{Lernstandsübersicht}\\            
        \vspace{0.5cm}
        \LARGE
        Mathematik
            
        \vspace{1.5cm}
            
        \textbf{Wissensklub GmbH}
            
        \vfill
            
        Klasse 10\\
        \textit{Gymnasium} Nordrhein-Westfalen
            
        \vspace{0.8cm}
            
        \includegraphics[width=0.5\textwidth]{Wissensklub-Logo.png}
            
        \Large
        2024          
    \end{center}
\end{titlepage}

\section{Daten und Wahrscheinlichkeit}
In diesem Kapitel wird die Grundlage für die Stochastik und Statistik gelegt. 
Den Schülern wird eine Sensibilität für das Lesen von Statistiken, z.B. anhand von Diagrammen, vermittelt.
Dabei werden Wahrscheinlichkeiten genauer untersucht, sowie bedingte Wahrscheinlichkeiten und die Stochastische Unabhängigkeit eingeführt.\\
\textbf{Voraussetzungen: }
\begin{itemize}
    \item \textbf{Brüche \& Anteile: } Anteile berechnen, als relative Häufigkeiten deuten können, Wahrscheinlichkeiten und relative Häufigkeiten unterscheiden
    \item \textbf{Wahrscheinlichkeit: } Wahrscheinlichkeiten schätzen, bei Laplace-Experimenten berechnen, Höufigkeiten vorhersagen
    \item \textbf{Zufallsexperimente: } Mehrstufige Zufallsexperimente, Baumdiagramme, Pfadregel für Wahrscheinlichkeiten 
\end{itemize}
\subsection{Inhalte}
\subsubsection*{Statistiken verstehen und beurteilen}
Je nachdem wie eine Statistik aufgebaut ist und ein zugehöriges Diagramm gestaltet wurde, kann man damit einen irreführenden Eindruck erwecken.
Hier werden die Skalierung der Achsen, die räumliche Darstellung im Diagramm, der Stickprobenumfang und vieles mehr überprüft um die Qualität statistischer Grafiken zu überprüfen.
Schüler sollen entscheiden, ob ein Diagramm passend gewählt und gestaltet wurde. Sind verpackte Aussagen auch mithilfe der Statistik zu unterstützen, oder manipulierend dargestellt?
\subsubsection*{Vierfeldertafel - mit Anteilen argumentieren}
Mit einer Vierfeldertafel lassen sich Statistiken in denen zwei Merkmale untersucht werden übersichtlich darstellen.
Die Struktur einer Vierfeldertafel und die Vorteile der Darstellung sollen vermittelt werden. 
Anhand der Merkmale sollen Vierfeldertafeln vervollständigt werden können, und das dem Sachkontext die richtigen Informationen eingetragen bzw. abgelesen werden können.
Es sollen die Wahrscheinlichkeiten bestimmter Ereignisse aus den Informationen der Vierfeldertafel berechnet werden.
\subsubsection*{Bedingte Wahrscheinlichkeiten}
Eng verwandt mit den Vierfeldertafeln ist das Konzept der bedingten Wahrscheinlichkeit.
Wie wahrscheinlich ist ein Ereignis, wenn wir wissen das ein bestimmtes Ereignis eintritt bzw. ein bestimmtes Merkmal vorliegt?
Die Schreibweise für bedingte Wahscheinlichkeiten sollen definiert werden.
Bedingte Wahrscheinlichkeiten sollen aus Vierfeldertafeln und Baumdiagrammen abgelesen werden können.
Es sollen Vierfeldertafeln in Baumdiagramme umgeschrieben werden können.
Die Formel für bedingte Wahrscheinlichkeiten soll hergeleitet werden und benutzt werden, um rechnerisch bedingte Wahrscheinlichkeiten auszurechnen.

\subsubsection*{Stochastische Unabhängigkeit}
Zwei Ereignisse sind stochastisch unabhängig, wenn die Tatsache das eines dieser Eregnisse eintritt nicht die Wahrscheinlichkeit bedingt, das das andere Ereignis eintritt.
Wahrscheinlichkeiten sollen unter der Annahme der stochastischen Unabhängigkeit berechnet werden.
Mit der Formel für die bedingte Wahrscheinlichkeit soll überprüft werden, ob zwei Eregnisse stochastisch unabhängig sind.
Im Sachkontext soll begründet werden, ob Ereignisse stochastisch unabhängig sein können.

\newpage
\section{Quadratische Gleichungen}
Quadratische Funktionen und Parabeln sind bereits bekannt. Wenn eine quadratische Funktion einen bestimmten Wert annehmen soll, kann man dies als quadratische Gleichung formulieren.
In diesem kapitel werden quadratische Gleichungen auf verschiedene Arten gelöst um damit Probleme zu lösen.\\
\textbf{Voraussetzungen}
\begin{itemize}
    \item \textbf{Lineare Funktionen \& Gleichungen: } Nullstellen lineare Funktionen, lineare Gleichungen mit Äquivalenzumformungen lösen, Schnittpunkte von Geraden bestimmen
    \item \textbf{Quadratische Funktionen: } grafisch darstellen, Quadratische Ergänzung durchführen
    \item \textbf{Quadratwurzel: } Berechnung von beliebigen Quadratwurzeln
\end{itemize}
\subsection{Inhalte}
\subsubsection*{Darstellungsformen quadratischer Funktionen}
Es werden die Begriffe Stauchung, Streckung und Scheitelpunkt- und Normalform wiederholt. Zusätzlich wird die faktorisierte Form anhand der Nullstellen eingeführt.
Die Vorteile der Darstellungen werden diskutiert, eine gegebene Funktion soll in die anderen Formen überführt werden könenn.
\subsubsection*{Quadratische Gleichungen grafisch lösen}
Quadratische Gleichungen sind solche, in denen eine quadratische Potenz $x^2$ in einer Variablen $x$ vorkommt. 
Diese Gleichungen können wir grafisch lösen, indem wir die Schnittpunkte einer Normalparabel mit einer linearen Funktion untersuchen.
Schüler sollen die quadratischen Gleichungen nach der Potenz $x^2$ umformen können. Die Normalparabel und die lineare Funktion auf der anderen Seite der Gleichung sollen in ein Koordinatensystem übertragen werden können.
Die Schnittpunkte sollen abgelesen werden und als Lösung der Gleichung interpretiert werden können.
\subsubsection*{Lösen einfacher quadratischer Gleichungen}
Wenn die quadratischen Gleichungen eine einfache Struktur haben, können wir diese auch ohne Zeichnung rechnerisch schnell lösen.
Hier sollen die Schüler Gleichungen faktorisieren können und die Wurzel ziehen können. Insbesondere soll entschieden werden ob eine Gleichung keine, eine oder zwei Lösungen hat, und in welchen Fällen dies eintritt.
\subsubsection*{Linearfaktorzerlegung}
Bei quadratischen Funktionen in der faktorisierten Form können wir die Nullstellen direkt ablesen.
Mit dem Satz von Vieta können wir die Nullstellen, und damit die faktorisierte Form, aus der Normalform direkt berechnen.
Hier sollen quadratische Funktionen in Linearfaktoren zerlegt werden. Dies kann durch geschicktes Ausprobieren oder mit dem Satz von Vieta geschehen.
Damit sollen quadratischen Gleichungen gelöst werden, nachdem  diese zunächst in die Normalform überführt wird.
\subsubsection*{Lösungsformel für quadratische Gleichungen}
Es wird die p-q-Formel für quadratische Gleichungen hergeleitet.
Mit dieser Formel sollen quadratische Gleichungen rechnerisch gelöst werden. 
Damit kann entschieden werden, wie viele Lösungen eine quadratische Gleichung hat.
Die Lösungen sollen aufgeschrieben werden und mit einer Probe überprüft werden können.
Der Zusammenhang zwischen Nullstellen einer quadratischen Funktion und der Lösung quadratischer Gleichungen soll hergestellt werden.
\subsubsection*{Modellierung mit quadratischen Gleichungen}
Es sollen systematisch Sachaufgaben bearbeitet werden, die mittels quadratischer Gleichungen modelliert und gelöst werden können.
Eine Strategie zur Lösung solcher Probleme soll erarbeitet und bei der Lösung von Aufgaben beibehaltet werden.
Der Rechenweg und Antworten sollen kritisch überprüft werden können. 
\newpage
\section{Ähnlichkeit}
Bisher wurden Figuren häufig auf ihre Struktur untersucht, z.B. rechtwinklige oder gleichsseitige Dreiecke. In diesem Kapitel werden Figure genauer durch ihre Größe unterschieden. Welche Figuren sind ähnlich, wie kann ich durch Streckung Figuren vergrößern oder verkleinern?
Die Strahlensätze werden eingeführt um Berechnungen von Längen durchzuführen
\textbf{Voraussetzungen: }
\begin{itemize}
    \item \textbf{Winkel: }  Erkennen und Bestimmen, Innenwinkelsätze von Vielecken
    \item \textbf{Kongruenz: } kongruente Figuren erkennen, Dreiecke mittels Kongruenzsätzen konstruieren
    \item \textbf{Gleichungen: } Einfache Gleichungen durch Äquivalenzumformungen lösen 
\end{itemize}
\subsection{Inhalte}
\subsubsection*{Zentrische Streckung}
Wie bei einer Projektion kann man ausgehend von einem Zentrum eine andere Figur in eine besimmte Richtung verschieben und um einen Streckfaktor vergrößern bzw. verkleinern.
Es soll verstanden werden, was bei postiven wie auch negativen Streckfaktoren geometrisch passiert.
Wie verändern sich die Längen von Strecken in einer Figur bei der zentrischen Streckung?
Ausgehen von zwei Figuren soll das Streckzentrum under Streckfaktor ermittelt werden.
Für eine Figur soll bei gegebenen Streckzentrum und -faktor eine zentrsiche Streckung ausgeführt werden.
\subsubsection*{Ähnlichkeit}
Wenn man Figuren durch eine zentrische Streckung in eine andere überführen kann, sind diese ähnlich.
Es soll erforscht werden, welche Merkmale eine Figur bei Ähnlichkeit erhalten bleiben und welche nicht.
Es werden Ähnlichkeitssätze für Dreiecke hergeleitet.
Anhand dieser Erkenntnisse sollen Figuren auf Ähnlichkeit untersucht werden.
Mithilfe der Ähnlichkeit sollen in einer Figur fehlende Seitenlängen ohne Nachmessen ermittelt werden.
\subsubsection*{Strahlensätze}
Mit den Strahlensätzen werden die konstant bleibenden Seitenverhältnisse bei der zentrischen Streckung formalisiert.
Bei passenden Figuren soll erkannt werden, wie die Strahlensätze eingesetzt werden können.
Bei einem Problem im Sachkontext sollen mithilfe der Strahlensätze fehlende Längen ermittelt werden.
Dabei müssen die Gleichungen der Strahlensätze entsprechend umgestellt werden, um fehlende Längen zu finden.
\newpage
\section{Exponentielles Wachstum}
Für exponentielles Wachstum werden die Besonderheiten und Unterschiede zu anderen Wachstumsformen erarbeitet.
exponentielle Gleichungen werden aufgestellt und mittels Logarithmen gelöst.
Probleme werden mithilfe der Exponentialfunktion modelliert.\\
\textbf{Voraussetzungen: }
\begin{itemize}
    \item \textbf{Lineare Funktionen: } Modellierung im Sachkontext
    \item \textbf{Prozente: } Prozentuelle Zu- oder Abnahme in einem Schritt berechnen
    \item \textbf{Zinsrechnung: } Wie entwickelt sich verzinstes Kapital über die Jahre? 
    \item \textbf{Wurzeln: } n-te Wurzel ziehen können um $x^n = a$ zu lösen
    \item \textbf{Potenzrechnung: } Potenzen mit natürlichem, ganzem oder rationalem Exponent berechnen
\end{itemize}
\subsection{Inhalte}
\subsubsection*{Exponentielles Wachstum}
Wenn in jedem Schritt ein \textit{Bestand} sich um denselben Faktor ändert, spricht man von exponentiellem Wachstum.
Aus dem Sachkontext, von einer Wertetabelle oder anhand einer Gleichung soll der Wachstumfaktor abgelesen werden
Eine Wertetabelle soll für ein vorliegenden exponentielles Wachstum vervollständigt werden.
Für einen bestimmten Zeitpunkt bzw. Zeitschritt soll der Bestand ausgerechnet werden können.
\subsubsection*{Exponentialfunktion}
Das exponentielle Wachstum wird verallgemeinert durch die Exponentialfunktion.
Die Begriffe Anfangswert, wachstumsfaktor und Verdopplungs- bzw. Halbierungszeit werden eingeführt.
Die Eigenschaften des Graphen einer Exponentialfunktion in Abhängigkeit des Wachstumsfaktors werden untersucht.
Eine Formel zur Bestimmung des Wachstumsfaktors anhand der Funtkionswerte wird hergeleitet.
Eine Gleichung der Exponentialfunktion anhand ihres Graphen soll hergeleitet werden.
\subsubsection*{Exponentialgleichungen und Logarithmen}
Der Logarithmus von b zur Basis a $log_a(b)$ ist die Lösung der Exponentialgleichung $a^x = b$.
Es wird erarbeitet, wie man Exponentialgleichungen mithilfe des Logarithmus lösen kann.
Einfache Logarithmen sollen im Kopf berechnet werden.
Rechenregeln für den Logarithmus werden hergeleitet und die Berechnung komplizierter Logarithmen mit dem Taschenrechner.
Exponentialgleichungen werden mithilfe des Logarithmus gelöst.
Der Logarithmus wird gelöst um Probleme zu exponentiellem Wachstum im Sachkontext zu lösen, z.B. um einen Zeitraum bei einem Wachstumsprozess zu finden.

\subsubsection*{Wachstumsprozesse modellieren}
Für verschiedene Wachstumsarten soll der passende Funktionstyp ausgesucht werden, z.B. lineare-, quadratische- oder Exponentialfunktionen.
Eine Gleichung für ein Modell soll bestimmt werden, z.B. anhand von gegebenen Funktionspunkten durch eine Beobachtung.
Bei ungenauen Messungen soll entschieden werden, wie gut eine Funktion ein Wachstum trifft und ob die Funktion eine gute Wahl ist.
Diese Strategien sollen genutzt werden um im Sachkontext ein Modell aufzustellen.
\newpage

\section{Trigonometrie}
In diesem Thema wird die Beziehung zwischen Seiten und Winkeln in Dreiecken intensiv untersucht.
Die trigonometrischen Funktionen werden eingeführt und die Verbindung zu rechtwinkligen Dreiecken hergestellt.
\textbf{Voraussetzungen: }
\begin{itemize}
    \item \textbf{Dreicke: } rechtwinklige Dreiecke, Satz des Pythagoras
    \item \textbf{Kongruenz \& Ähnlichkeit: } fehlende Seitenlängen berechnen bei ähnlichen Dreiecken
    \item \textbf{Gleichungen: } Lösen mittels Äquivalenzumformungen
\end{itemize}
\subsection{Inhalte}
\subsubsection*{Sinus und Kosinus}
Wenn man den Winkel $\alpha$ als Variable auffasst, lassen sich Sinus und Kosinus in Abhängigkeit von $\alpha$ als Funktionen $sin(\alpha), cos(\alpha)$ beschreiben.
Diese beiden Funktionen sollen benutzt werden um Berechnungen am rechtwinkligen Dreieck zu machen, und um fehlende Größen wie Winkel oder Seitenlängen zu finden.
Mithilfe des Innenwinkelsatzes sollen Ergebnisse kontrolliert werden.
Je nachdem ob Sinus oder Kosinus in einer Rechnung benutzt wurde, soll überlegt werden wie man die andere Funktion hätte nutzen können um auf das gleiche Ergebnis zu kommen.
\subsubsection*{Tangens}
Der Tangens gibt in Abhängigkeit des Winkels die \textit{Steigung} wieder.
Im rechtwinkligen Dreieck sollen mit dem Tangens wie im vorigen Kapitel Berechnungen zu fehlenden Größen bestehen.
Im Kontext von Steigungen sollen Sachaufgaben mit dem Tangens gelöst werden.
\subsubsection*{Problemlösen mit rechtwinkligen Dreiecken}
Zusammen mit dem Satz des Pythagoras, dem Erkennen von rechtwinkligen Dreiecken in anderen geometrischen Formen sollen  trigonometrische Funktionen genutzt werden um Probleme zu verstehen und zu lösen. 
\subsubsection*{Der Kosinussatz}
Den Satz des Pythagoras kann man nur in rechtwinkligen Dreiecken verwenden. Für beliebigen Dreiecke gilt der Kosinussatz.
Mit dem Kosinussatz sollen Seitenlängen und Winkelgrößen ausgerechnet werden.
Der Zusammenhang zwischen Kosinussatz und Satz des Pythagoras soll hergestellt werden.
Werte für den Kosinus zwischen $90°$ und $180°$ sollen mit einer Rechenregel $cos(\alpha) = - cos(180° - \alpha)$ berechnet werden.
\subsubsection*{Sinus und Kosinus am Einheitskreis}
Wenn wir am Rand des Einheitskreises Punkte einzeichnen, können wir gemeinsam mit der x-Achse ein rechtwinkliges Dreieck einzeichnen.
Anhand des Winkels im Kreis können wir die Koordinaten des Punktes direkt mit Sinus und Kosinus berechnen. 
Im Einheitskreis sind Sinus und Kosinugs jeweils entsprechend der Länge der Gegenkathete bzw. Ankathete.
Mithilfe des Einheitskreises sollen näherungsweise Sinus und Kosinus für einen gegebenen Winkel ausgerechnet werden.
Es soll untersucht werden, für welche Winkel der Sinus und Kosinus positiv oder negativ sind.
\subsubsection*{Die Sinusfunktion}
Wenn wir den Einheitskreis ablaufen und dabei die y-Werte betrachten, kriegen wir in Abhängigkeit des Winkels dene Graphen der Sinusfunktion.
Da Winkelgrößen schnell wachsen, betrachten wir die einzusetzenden Werte für die Sinusfunktion in Bogenmaß.
Die Schüler lernen es Winkel in Bogenmaß und Gradmaß ineinander umzurechnen.
Es wird erforscht, das die Sinusfunktion $360°$ bzw. $2\pi$-periodisch ist.

Es sollen näherungsweise Werte am Graphen der Sinusfunktion abgelesen werden.
Für bestimmte Gleichungen der Sinusfunktion, z.B. $sin(\alpha) = 0.5$ sollen alle Werte für $\alpha$ gefunden werden, sodass die Gleichung stimmt.
\subsubsection*{Transformationen der Sinusfunktion}
Hier werden Streckungen und Stachungen in x- und y-Richtung erarbeitet.
Es werden die Begriffe Amplitude und Periode der Sinusfunktion eingeführt.
Zu verschiedenen Graphen transformierter Sinusfunktionen soll die richtige Funktionsvorschrift zugeordnet werden können.
Für eine Funktionsvorschrift soll die Amplitude und Periode der Funktion angegeben werden.
\subsubsection*{Beschreibung periodischer Vorgänge}
Daten in einer Statistik können periodisch ablaufen.
Hier lernen die Schüler Daten auszuwerten und zu interpretieren.
Es wird eine Modellfunktion anhand einer abgeschätzten Amplitude, Periode und Verschiebung aufgestellt.
Strategien zum Ausrechnen dieser Kennwerte werden erarbeitet.
Abweichungen und die Entwicklung der Funktion werden genutzt um zu beurteilen wie passend die Modellfunktion ist.
\end{document}