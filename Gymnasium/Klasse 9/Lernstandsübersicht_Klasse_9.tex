\documentclass{article}
\usepackage[ngerman]{babel}
\usepackage[margin=2cm]{geometry}
\usepackage{amsmath, amssymb, amsfonts}
\usepackage{tikz}
\usepackage{graphicx}
\usepackage{fancyhdr}

\usepackage{tcolorbox} %Umgebung mit farbigem Hintergrund

\graphicspath{{../../}}
%Seitenstil modifizieren
\pagestyle{fancy}% eigenen Seitestil aktivieren}
\fancyhf{}% Alle Felder loeschen

\fancyhead[L]{
\begin{tabular}[b]{l}
Lernstandsübersicht\\
Mathematik\\
Klasse 9
\end{tabular}}
\fancyhead[R]{
\includegraphics[height=3\baselineskip]{Wissensklub-Logo.png}
}
\addtolength{\headheight}{2\baselineskip}
\addtolength{\headheight}{0.61pt}
\fancyhead[C]{2024}

\begin{document}
\begin{titlepage}
    \begin{center}
        \vspace*{1cm}
            
        \Huge
        \textbf{Lernstandsübersicht}\\            
        \vspace{0.5cm}
        \LARGE
        Mathematik
            
        \vspace{1.5cm}
            
        \textbf{Wissensklub GmbH}
            
        \vfill
            
        Klasse 9\\
        \textit{Gymnasium} Nordrhein-Westfalen
            
        \vspace{0.8cm}
            
        \includegraphics[width=0.5\textwidth]{Wissensklub-Logo.png}
            
        \Large
        2024          
    \end{center}
\end{titlepage}

\section{Reelle Zahlen}
Die bekannten Zahlen werden um die irrationalen Zahlen zu den reellen Zahlen erweitert.\\
\textbf{Voraussetzungen:}
\begin{itemize}
    \item Rechnen mit rationalen Zahlen (Klasse 7)
    \item Fläche von Quadraten berechnen (Klasse 5, 6)
    \item Potenzen (Klasse 5, 6)
\end{itemize}
\subsection{Inhalte}
\subsubsection*{Quadratwurzeln}
\subsubsection*{Wurzeln näherungsweise bestimmen}
\subsubsection*{Irrationale Zahlen}
\subsubsection*{Geschickt mit Wurzeln rechnen}
\newpage
\section{Quadratische Funktionen}
Als wichtigen Typ von Funktionen werden quadratische Funktionen grafisch und algebraisch eingeführt.\\
\textbf{Voraussetzungen:}
\begin{itemize}
    \item Funktionen grafisch zeichnen, Wertetabellen aufstellen und ablesen und Funktionsgleichungen aufstellen
    \item Lineare Funktionen und Gleichungssysteme
    \item binomische Formeln
    \item Funktionen im Sachkontext einsetzen
\end{itemize}
\subsection{Inhalte}
\subsubsection*{$f(x) = a x^2$}
\subsubsection*{Scheitelpunktform quadratischer Funktionen}
\subsubsection*{Normalform und quadratische Ergänzung}
\subsubsection*{Aufstellen quadratischer Funtkionsgleichungen}
\newpage
\section{Kreise, Prismen und Zylinder}
\subsection{Inhalte}
\subsubsection*{Kreisumfang und Kreisfläche}
\subsubsection*{Kreisteile}
\subsubsection*{Flächen bei Prismen und Zylindern}
\subsubsection*{Volumen von Prismen und Zylindern}
\newpage
\section{Potenzen und Potenzgesetze}
\subsubsection*{Potenzen mit ganzzahligen Exponenten}
\subsubsection*{Zahlen mit Zehnerpotenzen schreiben}
\subsubsection*{Potenzen mit gleicher Basis}
\subsubsection*{Potenzen mit gleichem Exponenten}
\subsubsection*{Potenzieren von Potenzen}
\subsubsection*{Potenzen mit rationalen Exponenten}
\newpage
\section{Satz des Pythagoras und Körper}
\subsection{Inhalte}
\subsubsection*{Der Satz des Pythagoras}
\subsubsection*{Pythagoras in Figuren und Körpern}
\subsubsection*{Pyramiden}
\subsubsection*{Kegel}
\subsubsection*{Kugeln}
\end{document}